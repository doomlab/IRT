\documentclass[english,man, mask]{apa6}

\usepackage{amssymb,amsmath}
\usepackage{ifxetex,ifluatex}
\usepackage{fixltx2e} % provides \textsubscript
\ifnum 0\ifxetex 1\fi\ifluatex 1\fi=0 % if pdftex
  \usepackage[T1]{fontenc}
  \usepackage[utf8]{inputenc}
\else % if luatex or xelatex
  \ifxetex
    \usepackage{mathspec}
    \usepackage{xltxtra,xunicode}
  \else
    \usepackage{fontspec}
  \fi
  \defaultfontfeatures{Mapping=tex-text,Scale=MatchLowercase}
  \newcommand{\euro}{€}
\fi
% use upquote if available, for straight quotes in verbatim environments
\IfFileExists{upquote.sty}{\usepackage{upquote}}{}
% use microtype if available
\IfFileExists{microtype.sty}{\usepackage{microtype}}{}

% Table formatting
\usepackage{longtable, booktabs}
\usepackage{lscape}
% \usepackage[counterclockwise]{rotating}   % Landscape page setup for large tables
\usepackage{multirow}		% Table styling
\usepackage{tabularx}		% Control Column width
\usepackage[flushleft]{threeparttable}	% Allows for three part tables with a specified notes section
\usepackage{threeparttablex}            % Lets threeparttable work with longtable

% Create new environments so endfloat can handle them
% \newenvironment{ltable}
%   {\begin{landscape}\begin{center}\begin{threeparttable}}
%   {\end{threeparttable}\end{center}\end{landscape}}

\newenvironment{lltable}
  {\begin{landscape}\begin{center}\begin{ThreePartTable}}
  {\end{ThreePartTable}\end{center}\end{landscape}}

  \usepackage{ifthen} % Only add declarations when endfloat package is loaded
  \ifthenelse{\equal{\string man, mask}{\string man}}{%
   \DeclareDelayedFloatFlavor{ThreePartTable}{table} % Make endfloat play with longtable
   % \DeclareDelayedFloatFlavor{ltable}{table} % Make endfloat play with lscape
   \DeclareDelayedFloatFlavor{lltable}{table} % Make endfloat play with lscape & longtable
  }{}%



% The following enables adjusting longtable caption width to table width
% Solution found at http://golatex.de/longtable-mit-caption-so-breit-wie-die-tabelle-t15767.html
\makeatletter
\newcommand\LastLTentrywidth{1em}
\newlength\longtablewidth
\setlength{\longtablewidth}{1in}
\newcommand\getlongtablewidth{%
 \begingroup
  \ifcsname LT@\roman{LT@tables}\endcsname
  \global\longtablewidth=0pt
  \renewcommand\LT@entry[2]{\global\advance\longtablewidth by ##2\relax\gdef\LastLTentrywidth{##2}}%
  \@nameuse{LT@\roman{LT@tables}}%
  \fi
\endgroup}


\ifxetex
  \usepackage[setpagesize=false, % page size defined by xetex
              unicode=false, % unicode breaks when used with xetex
              xetex]{hyperref}
\else
  \usepackage[unicode=true]{hyperref}
\fi
\hypersetup{breaklinks=true,
            pdfauthor={},
            pdftitle={Does the Delivery Matter? Examining Randomization at the Item Level},
            colorlinks=true,
            citecolor=blue,
            urlcolor=blue,
            linkcolor=black,
            pdfborder={0 0 0}}
\urlstyle{same}  % don't use monospace font for urls

\setlength{\parindent}{0pt}
%\setlength{\parskip}{0pt plus 0pt minus 0pt}

\setlength{\emergencystretch}{3em}  % prevent overfull lines

\ifxetex
  \usepackage{polyglossia}
  \setmainlanguage{}
\else
  \usepackage[english]{babel}
\fi

% Manuscript styling
\captionsetup{font=singlespacing,justification=justified}
\usepackage{csquotes}
\usepackage{upgreek}

 % Line numbering
  \usepackage{lineno}
  \linenumbers


\usepackage{tikz} % Variable definition to generate author note

% fix for \tightlist problem in pandoc 1.14
\providecommand{\tightlist}{%
  \setlength{\itemsep}{0pt}\setlength{\parskip}{0pt}}

% Essential manuscript parts
  \title{Does the Delivery Matter? Examining Randomization at the Item Level}

  \shorttitle{Item Randomization}


  \author{Erin M. Buchanan\textsuperscript{1}, Riley E. Foreman\textsuperscript{1}, Becca N. Johnson\textsuperscript{1}, Jeffrey M. Pavlacic\textsuperscript{2}, Rachel L. Swadley\textsuperscript{1}, \& Stefan E. Schulenberg\textsuperscript{2}}

  % \def\affdep{{"", "", "", "", "", ""}}%
  % \def\affcity{{"", "", "", "", "", ""}}%

  \affiliation{
    \vspace{0.5cm}
          \textsuperscript{1} Missouri State University\\
          \textsuperscript{2} University of Mississippi  }

  \authornote{
    Erin M. Buchanan is an Associate Professor of Quantitative Psychology at
    Missouri State University. Riley E. Foreman received his undergraduate
    degree in Psychology and Cell and Molecular Biology at Missouri State
    University and is currently at Kansas City University of Medicine and
    Biosciences. Becca N. Johnson is a masters degree candidate at Missouri
    State University. Jeffrey M. Pavlacic is a doctoral candidate at The
    University of Mississippi. Rachel N. Swadley completed her master's
    degree in Psychology at Missouri State University. Stefan E. Schulenberg
    is a Professor of Clinical Psychology at The University of Mississippi
    and Director of the Clinical Disaster Research Center. On behalf of all
    authors, the corresponding author states that there is no conflict of
    interest.
    
    Correspondence concerning this article should be addressed to Erin M.
    Buchanan, 901 S. National Ave. E-mail:
    \href{mailto:erinbuchanan@missouristate.edu}{\nolinkurl{erinbuchanan@missouristate.edu}}
  }


  \abstract{Scales that are psychometrically sound, meaning those that meet
established standards regarding reliability and validity when measuring
one or more constructs of interest, are customarily evaluated based on a
set modality (i.e., computer or paper) and administration (fixed-item
order). Deviating from an established administration profile could
result in non-equivalent response patterns, indicating the possible
evaluation of a dissimilar construct. Randomizing item administration
may alter or eliminate these effects. Therefore, we examined the
differences in scale relationships for randomized and nonrandomized
computer delivery for two scales measuring meaning/purpose in life.
These scales have questions about suicidality, depression, and life
goals that may cause item reactivity (i.e.~a changed response to a
second item based on the answer to the first item). Results indicated
that item randomization does not alter scale psychometrics for meaning
in life scales, which implies that results are comparable even if
researchers implement different delivery modalities.}
  \keywords{scales, randomization, item analysis \\

    
  }





\usepackage{amsthm}
\newtheorem{theorem}{Theorem}
\newtheorem{lemma}{Lemma}
\theoremstyle{definition}
\newtheorem{definition}{Definition}
\newtheorem{corollary}{Corollary}
\newtheorem{proposition}{Proposition}
\theoremstyle{definition}
\newtheorem{example}{Example}
\theoremstyle{definition}
\newtheorem{exercise}{Exercise}
\theoremstyle{remark}
\newtheorem*{remark}{Remark}
\newtheorem*{solution}{Solution}
\begin{document}

\maketitle

\setcounter{secnumdepth}{0}



The use of the Internet has been integrated into daily life as a means
of accessing information, interacting with others, and tending to
required tasks. The International Telecommunication Union reports that
over half the world is online, and 70\% of 15-24 year olds are on the
internet (Sanou, 2017). Further, the Nielson Total Audience report from
2016 indicates that Americans spend nearly 11 hours a day in media
consumption (Media, 2016). Researchers discovered that online data
collection can be advantageous over laboratory and paper data
collection, as it is often cheaper and more efficient (Ilieva, Baron, \&
Healy, 2002; Reips, 2012; Schuldt \& Totten, 1994). Internet
questionnaires first appeared in the early 90s when HTML scripting code
integrated form elements, and the first experiments appeared soon after
(Musch \& Reips, 2000; Reips, 2002). The first experimental lab on the
internet was the Web Experimental Psychology Lab formed by Reips
(\url{http://www.wexlab.eu}), and the use of the Internet to collect
data has since grown rapidly (Reips, 2002). What started with email and
HTML forms has since moved to whole communities of available
participants including websites like Amazon's Mechanical Turk and
Qualtrics' Participant Panels. Participants of all types and forms are
easily accessible for somewhat little to no cost.

Our ability to collect data on the Internet has inevitably lead to the
question of measurement invariance between in person and online data
collection methods (Buchanan et al., 2005; Meyerson \& Tryon, 2003).
Invariance implies that different forms, data collection procedures, or
even target demographics produce comparable sets of responses, which is
a desirable characteristic to ensure a minimal number of confounding
variables (Brown, 2006). According to Deutskens, Ruyter, and Wetzels
(2006), mail surveys and online surveys produce nearly identical results
regarding the accuracy of the data collected online versus by mail. Only
minor differences arise between online surveys and mail in surveys when
it comes to participant honesty and suggestions. For example,
participants who responded to surveys online provided more suggestions,
lengthier answers, and greater information about competitors in the
field that they may prefer (Deutskens et al., 2006). The hypothesis as
to why individuals may be more honest online than in person is that the
individual may feel more anonymity and less social desirability effects
due to the nature of the online world, therefore less concerned about
responding in a socially polite way (Joinson, 1999). A trend found by
Fang, Wen, and Pavur (2012a) shows individuals are more likely to
respond to surveys online with extreme scores, rather than mid-range
responses on Likert scales due to the lessened social desirability
factor. There may be slight cultural differences in responses online.
For example, collectivistic cultures showed greater tendency toward
mid-range responses on Likert scales via in-person and online due to
placing greater value on how they are socially perceived; however, the
trend is still the same as scores are more extreme online versus in
person or by mail (Fang, Wen, \& Prybutok, 2012b).

Although work by Dillman and his group (Dillman, Smyth, \& Christian,
2008; Frick, Bächtiger, \& Reips, 2001; Smyth, 2006), among others, has
shown that many web surveys are plagued by problems of usability,
display, coverage, sampling, non-response, or technology, other studies
have found internet data to be reliable and almost preferable as it
produces a varied demographic response compared to the traditional
sample of introduction to psychology college students while also
maintaining data equivalence (Lewis, Watson, \& White, 2009). However,
equivalence in factor structure may be problematic, as Buchanan et al.
(2005) have shown that factor structure was not replicable in online and
in person surveys. Other work has shown equivalence using a comparison
of correlation matrices (Meyerson \& Tryon, 2003) or \emph{t}-tests
(Schulenberg \& Yutrzenka, 1999, 2001), and the literature is mixed on
how different methodologies impact factor structure. Weigold, Weigold,
and Russell (2013) recently examined both quantitative and research
design questions (i.e.~missing data) on Internet and paper-and-pencil
administration which showed that the administrations were generally
equivalent for quantitative structure but research design issues showed
non-equivalence. Other potential limitations to online surveys include
the accessibility of different populations to the Internet (Frick et
al., 2001), selection bias (Bethlehem, 2010), response rates (Cantrell
\& Lupinacci, 2007; Cook, Heath, \& Thompson, 2000; De Leeuw \& Hox,
1988; Hox \& De Leeuw, 1994), attrition (Cronk \& West, 2002), and
distraction (Tourangeau, Rips, \& Rasinski, 1999). Many of these
concerns have been alleviated in the years since online surveys were
first developed, especially with the advent of panels and Mechanical
Turk to reach a large, diverse population of participants (Buhrmester,
Kwang, \& Gosling, 2011).

With the development of advanced online survey platforms such as
Qualtrics and Survey Monkey, researchers have the potential to control
potentially confounding research design issues through randomization,
although other issues may still be present, such as participant
misbehavior (Nosek, Banaji, \& Greenwald, 2002). Randomization has been
a hallmark of good research practice, as the order or presentation of
stimuli can be a noise variable in a study with multiple measures
(Keppel \& Wickens, 2004). Thus, researchers have often randomized
scales by rotating the order of presentation in paper format or simply
clicking the randomization button for web-based studies. This practice
has counterbalanced out any order effects of going from one scale to the
next (Keppel \& Wickens, 2004). However, while scale structure has
remained constant, these items are still stimuli within a larger
construct. Therefore, these construct-related items have the ability to
influence the items that appear later on the survey, which we call item
reactivity. For example, a question about being \emph{prepared for
death} or \emph{thoughts about suicide} might change the responses to
further questions, especially if previous questions did not alert
participants to be prepare for that subject matter.

Scale development typically starts with an underlying latent variable
that a researcher wishes to examine through measured items or questions
(DeVellis, 2016). Question design is a well-studied area that indicates
that measurement is best achieved through questions that are direct,
positively worded, and understandable to the subject (Dillman et al.,
2008). Olson (2010) suggests researchers design a multitude of items in
order to investigate and invite subject matter experts to examine these
questions. Subject matter experts were found to be variable in their
agreement, but excellent at identifying potentially problematic
questions. After suggested edits from these experts, a large sample of
participant data is collected. While item response theory is gaining
traction, classical test theory has dominated this area through the use
of exploratory and confirmatory factor analysis (EFA, CFA; Worthington
\& Whittaker, 2006). EFA elucidates several facets of how the measured
items represent the latent trait through factor loadings and overall
model fit (Tabachnick \& Fidell, 2012). Factor structure represents the
correlation between item scores and factors, where a researcher wishes
to find items that are strongly related to latent traits. Items that are
not related to the latent trait, usually with factor loadings below .300
(Preacher \& MacCallum, 2003) are discarded. Model fit is examined when
simple structure has been achieved (i.e.~appropriate factor loadings for
each item), and these fit indices inform if the items and factor
structure model fit the data well. Well-designed scales include items
that are highly related to their latent trait and have excellent fit
indices. Scale development additionally includes the examination of
other measures of reliability (alpha) and construct validity (relation
to other phenomena) but the focus of the scale shifts to subscale or
total scores (Buchanan, Valentine, \& Schulenberg, 2014). Published
scales are then distributed for use in the form that is presented in the
publication, as item order is often emphasized through important notes
about reverse scoring and creating subscale scores.

The question is no longer whether web-based surveys are reliable sources
of data collection; the theory now is in need of a shift to whether or
not item-randomization in survey data collection creates psychometric
differences. These scale development procedures focus on items, and
EFA/CFA statistically try to mimic variance-covariance structure by
creating models of the data with the same variance-covariance matrix. If
we imagine that stimuli in a classic experimental design can influence
the outcome of a study because of their order, then certainly the
stimuli on a scale (i.e., the items) can influence the pattern of
responses for items. This area of study is relatively unexplored, as
easily randomizing items has only recently become available for
researchers.

Therefore, this study focuses on potential differences in results based
on item randomization delivery methodology. The current project examined
large samples on two logotherapy-related scales, as these scales include
potentially reactive items, as well as both a dichotomous True/False and
traditional Likert format for the same items. Large samples were
desirable to converge on a stable, representative population; however,
false positives (i.e., Type I errors) can occur by using large \emph{N}.
Recent developments in the literature focusing on null hypothesis
testing make it especially important to present potential alternatives
to \emph{p}-values (Valentine, Buchanan, Scofield, \& Beauchamp, 2017).
While a large set of researchers have argued that the literature is full
of Type I errors (Benjamin et al., 2018), and thus, the \(\alpha\) value
should be shifted lower (i.e., \emph{p} \textless{} .005 for statistical
significance), an equally large set of researchers counter this argument
as unfounded and weak (Lakens et al., n.d.). We provide multiple sources
of evidence (\emph{p}-values, effect sizes, Bayes Factors, and tests of
equivalence) to determine if differences found are not only
statistically significant, but also practically significant. In our
study, we expand to item randomization for online based surveys,
examining the impact on factor loadings, variance-covariance structure,
item means, and total scores again providing evidence of
difference/non-difference from multiple statistical sources. Finally, we
examine these scenarios with a unique set of scales that have both
dichotomous True/False and traditional Likert formats to explore how the
answer response options might impact any differences found between
randomized and nonrandomized methodologies.

\section{Method}\label{method}

\subsection{Participants}\label{participants}

The sample population consisted of undergraduate students at a large
Midwestern University, placing the approximate age of participants at
around 18-22. Table \ref{tab:demo-table} includes the demographic
information about all datasets. Only two scales were used from each
dataset, as described below. Participants were generally enrolled in an
introductory psychology course that served as a general education
requirement for the university. As part of the curriculum, the students
were encouraged to participate in psychology research programs,
resulting in their involvement in this study. These participants were
given course credit for their participation.

\subsection{Materials}\label{materials}

Of the surveys included within each larger study, two questionnaires
were utilized: the Purpose in Life Questionnaire (PIL; Crumbaugh \&
Maholick, 1964) and the Life Purpose Questionnaire (LPQ; Hutzell, 1988).

\subsubsection{The Purpose in Life
Questionnaire}\label{the-purpose-in-life-questionnaire}

The PIL is a 20-item questionnaire that assesses perceived meaning and
life purpose. Items are structured in a 7-point Likert type response
format; however, each item has different anchoring points that focus on
item content. Total scores are created by summing the items, resulting
in a range of 20 to 140 for the overall score. The reliability for the
scale is generally high, ranging from .70 to .90 (Schulenberg, 2004;
Schulenberg \& Melton, 2010). Previous work on validity for the PIL
showed viable one- and two-factor models, albeit factor loadings varied
across publications (see Schulenberg \& Melton, 2010 for a summary), and
these fluctuating results lead to the development of a 4-item PIL short
form (Schulenberg, Schnetzer, \& Buchanan, 2011).

\subsubsection{Life Purpose
Questionnaire}\label{life-purpose-questionnaire}

The LPQ was modeled after the full 20-item PIL questionnaire, also
measuring perceived meaning and purpose in life. The items are
structured in a true/false response format, in contrast to the Likert
response format found on the PIL. Each question is matched to the PIL
with the same item content, altering the question to create binary
answer format. After reverse coding, zero on an item would indicate low
meaning, while one on an item would indicate high meaning. A total score
is created by summing item scores, resulting in a range from 0 to 20. In
both scales, higher scores indicated greater perceived meaning in life.
Reliability for this scale is also correspondingly high, usually in the
.80 range (Melton \& Schulenberg, 2008; Schulenberg, 2004).

These two scales were selected because they contained the same item
content with differing response formats, which would allow for cross
comparisons between results for each scale.

\subsection{Procedure}\label{procedure}

The form of administration was of interest to this study, and therefore,
two formats were included: computerized administration in nonrandom
order and computerized administration with a randomized question order.
Computerized questionnaires were available for participants to access
electronically, and they were allowed to complete the experiment from
anywhere with the Internet through Qualtrics. To ensure participants
were properly informed, both an introduction and a debriefing were
included within the online form. Participants were randomly assigned to
complete a nonrandomized or randomized version of the survey.
Nonrandomized questionnaires followed the original scale question order,
consistent with paper delivery format. A different group of participants
were given each question in a randomized order within each scale
(i.e.~all PIL and LPQ questions will still grouped together on one
page). The order of administration of the two scales was randomized
across participants for both groups. Once collected, the results were
then amalgamated into a database for statistical analysis.

\section{Results}\label{results}

\subsection{Hypothesis and Data-Analytic
Plan}\label{hypothesis-and-data-analytic-plan}

Computer forms were analyzed by randomized and nonrandomized groups to
examine the impact of randomization on covariance structure, factor
loadings, item means, and total scores. We expected to find that these
forms may potentially vary across covariance structure and item means,
which would indicate differences in reactivity to questions (i.e.~item
four always has item three as a precursor on a nonrandom form, while
item four may have a different set of answers when prefaced with other
questions). Factor loadings were assessed to determine if differences in
randomization caused a change in focus, such that participant
interpretation of the item changed the relationship to the latent
variable. However, we did not predict if values would change, as latent
trait measurement should be consistent. Last, we examined total scores;
however, it was unclear if these values would change. A difference in
item means may result in changes in total scores, but may also result in
no change if some item means decrease, while others increase.

Each hypothesis was therefore tested using four dependent measures.
First, we examined the variance-covariance matrix for each type of
delivery and compared the matrices to each other by using root mean
squared error (RMSE). RMSE estimates the difference between covariance
matrices and is often used in structural equation modeling to determine
if models have good fit to the data. A criterion of \textless{} .06 for
good fit, .06-.08 for acceptable fit, and \textgreater{} .10 for bad fit
was used (Hu \& Bentler, 1999). This analysis was used to determine if
the change in delivery changed the structure of the item relationships
to each other (i.e.~if the covariance matrices are different). RMSE
values were calculated using the \emph{monomvn} package in \emph{R}
(Gramacy \& Pantaleoy, 2010).

We then conducted an exploratory factor analysis on both scales using
one-factor models to examine the loading of each item on its latent
trait. The PIL factor structure is contested (Strack \& Schulenberg,
2009) with many suggestions as to latent structure for one- and
two-factor models. The LPQ has seen less research on factor structure
(Schulenberg, 2004). This paper focused on loadings on one global latent
trait to determine if the manipulation of delivery impacted factor
loadings. We used a one-factor model and included all questions to focus
on the loadings, rather than the factor structure. The analysis was
performed using the \emph{psych} package in \emph{R} with maximum
likelihood estimation and an oblique (oblimin) rotation. The LPQ factor
analysis used tetrachoric correlation structure to control for the
dichotomous format of the scale, rather than traditional Pearson
correlation structure. The loadings were then compared using a matched
dependent \emph{t}-test (i.e.~item one to item one, item two to item
two) to examine differences between nonrandomized and randomized
computer samples.

Next, item averages were calculated across all participants for each
item. These 20 items were then compared in a matched dependent
\emph{t}-test to determine if delivery changed the mean of the item on
the PIL or LPQ. While covariance structure elucidates the varying
relations between items, we may still find that item averages are pushed
one direction or another by a change in delivery and still maintain the
same correlation between items. If this test was significant, we
examined the individual items across participants for large effect
sizes, as the large sample sizes in this study would create significant
\emph{t}-test follow ups.

Last, the total scores for each participant were compared across
delivery type using an independent \emph{t}-test. Item analyses allow a
focus on specific items that may show changes, while total scores allow
us to investigate if changes in delivery alter the overall score that is
used in other analyses or possible clinical implications. For analyses
involving \emph{t}-tests, we provide multiple measures of evidentiary
value so that researchers can weigh the effects of randomization on
their own criterion. Recent research on \(\alpha\) criteria has shown
wide disagreement on the usefulness of \emph{p}-values and set cut-off
scores (Benjamin et al., 2018; Lakens et al., n.d.). Therefore, we
sought to provide traditional null hypothesis testing results
(\emph{t}-tests, \emph{p}-values) and supplement these values with
effect sizes (Buchanan, Valentine, \& Scofield, 2017; \emph{d} and
non-central confidence intervals, Cumming, 2014; Smithson, 2001), Bayes
Factors ({\textbf{???}}; Morey \& Rouder, 2015), and one-sided tests of
equivalence (TOST, Cribbie, Gruman, \& Arpin-Cribbie, 2004; Lakens,
2017; Rogers, Howard, \& Vessey, 1993; Schuirmann, 1987). We used the
average standard deviation of each group as the denominator for \emph{d}
calculation as follows:

\[
d_{av} = \frac {(M_1 -  M_2) } { \frac{SD_1 + SD_2 } {2} }
\] This effect size is less biased than the traditional \(d_z\) formula,
wherein mean differences are divided by the standard deviation of the
difference scores (Lakens, 2013). The difference scores standard
deviation is often much smaller than the average of the standard
deviations of each level, which can create an upwardly biased effect
size (Cumming, 2014). This bias can lead researchers to interpret larger
effects for a psychological phenomenon than actually exist.

Bayes Factors are calculated in opposition to a normal frequentist
(NHST) approach, as a ratio of the likelihood of two models. Traditional
NHST focuses on the likelihood of the data, given the null hypothesis is
true, and Bayesian analysis instead posits the likelihood of a
hypothesis given the data. Prior distributions are our estimation of the
likelihood of our hypothesis before the data was collected, which is
combined with the data collected to form a posterior belief of our
hypothesis. We chose to use Bayes Factors as a middle ground to the
Bayesian analysis continuum, that uses mildly uninformative priors and
allows for the data to strongly impact the posterior distribution. The
choice of prior distribution can heavily influence the posterior belief,
in that uninformative priors allow the data to comprise the posterior
distribution. However, most researchers have a background understanding
of their field, thus, making completely uninformative priors a tenuous
assumption. Because of the dearth of literature in this field, there is
not enough previous information to create a strong prior distribution,
which would suppress the effect of the data on posterior belief. The
\emph{BayesFactor} package (Morey \& Rouder, 2015) uses recommended
default priors that cover a wide range of data (Morey \& Rouder, 2015;
Rouder, Speckman, Sun, Morey, \& Iverson, 2009) of a Jeffreys prior with
a fixed rscale (0.5) and random rscale (1.0). The alternative model is
generally considered a model wherein means between groups or items
differ, and this model is compared to a null model of no mean
differences. The resulting ratio is therefore the odds of the
alternative model to the null, where BF values less than one indicate
evidence for the null, values at one indicate even evidence for the null
and alternative, and values larger than one indicate evidence for the
alternative model. While some researchers have posed labels for BF
values ({\textbf{???}}), we present these values as a continuum to allow
researchers to make their own decisions (Morey, 2015; Morey \& Rouder,
2015).

NHST has also been criticized for an inability to test the null
hypothesis, and thus, show evidence of the absence of an effect.
Non-significant \emph{p}-values are often misinterpreted as evidence for
the null hypothesis (Lakens, 2017). However, we can use the traditional
frequentist approach to determine if an effect is within a set of
equivalence bounds. We used the two one-sided tests approach to specify
a range of raw-score equivalence that would be considered supportive of
the null hypothesis (i.e.~no worthwhile effects or differences). TOST
are then used to determine if the values found are outside of the
equivalence range. Significant TOST values indicate that the effects are
\emph{within} the range of equivalence. We used the \emph{TOSTER}
package (Lakens, 2017) to calculate these values, and graphics created
from this package can be found online at \url{https://osf.io/gvx7s/}.

The equivalence ranges are often tested by computing an expected effect
size of negligible range; however, the TOST for dependent \emph{t} uses
\(d_z\), which can overestimate the effect size of a phenomena (Cumming,
2014; Lakens, 2013). Therefore, we calculated TOST tests on raw score
differences to alleviate the overestimation issues. For EFA, we used a
change score of .10 in the loadings, as Comrey and Lee (1992) suggested
loading estimation ranges, such as .32 (poor) to .45 (fair) to .55
(good), and the differences in these ranges are approximately .10 (as
cited in Tabachnick \& Fidell, 2012, p. 654). Additionally, this score
would amount to a small correlation change using traditional guidelines
for interpretation of \emph{r} (Cohen, 1992). For item and total score
differences, we chose a 5\% change in magnitude as the raw score cut off
as a modest raw score change. To calculate that change for total scores,
we used the following formula:

\[
(Max*N_{Questions} - Min*N_{Questions}) * Change
\] Minimum and maximum values indicate the lower and upper end of the
answer choices (i.e.~1 and 7), and change represented the proportion
magnitude change expected. Therefore, for total PIL scores, we proposed
a change in 6 points to be significant, while LPQ scores would change 1
point to be a significant change. For item analyses, we divided the
total score change by the number of items to determine how much each
item should change to impact the total score a significant amount (PIL =
0.30, LPQ = .05).

\subsection{Data Screening}\label{data-screening}

Each dataset was analyzed separately by splitting on scale and
randomization, and first, all data were screened for accuracy and
missing data. Participants with more than 5\% missing data (i.e.~2 or
more items) were excluded. Data were imputed using the \emph{mice}
package in \emph{R} for participants with less than 5\% of missing data
(Van Buuren \& Groothuis-Oudshoorn, 2011). Next, each dataset was
examined for multivariate outliers using Mahalanobis distance
(Tabachnick \& Fidell, 2012). Each dataset was then screened for
multivariate assumptions of additivity, linearity, normality,
homogeneity, and homoscedasticity. While some data skew was present,
large sample sizes allowed for the assumption of normality of the
sampling distribution. Information about the number of excluded data
points in each step is presented in Table \ref{tab:demo-table}.

\begin{table}[tbp]
\begin{center}
\begin{threeparttable}
\caption{\label{tab:demo-table}Demographic and Data Screening Information}
\begin{tabular}{lcccccc}
\toprule
Group & \multicolumn{1}{c}{Female} & \multicolumn{1}{c}{White} & \multicolumn{1}{c}{Age (SD)} & \multicolumn{1}{c}{Original N} & \multicolumn{1}{c}{Missing N} & \multicolumn{1}{c}{Outlier N}\\
\midrule
PIL Random & 61.6 & 81.1 & 19.50 (2.93) & 1462 & 333 & 58\\
PIL Not Random & 54.1 & 78.6 & 19.68 (3.58) & 915 & 51 & 36\\
LPQ Random & - & - & - & 1462 & 555 & 23\\
LPQ Not Random & - & - & - & 915 & 150 & 15\\
\bottomrule
\addlinespace
\end{tabular}
\begin{tablenotes}[para]
\textit{Note.} Participants took both the PIL and LPQ scale, therefore, random and not random demographics are the same. Not every participant was given the LPQ, resulting in missing data for those subjects. Several PIL participants were removed because they were missing an item on their scale.
\end{tablenotes}
\end{threeparttable}
\end{center}
\end{table}

\subsection{PIL Analyses}\label{pil-analyses}

\subsubsection{Covariance Matrices}\label{covariance-matrices}

Covariance structure was considered different (i.e.~above .10; Hu \&
Bentler, 1999) for the randomized and not randomized forms of item
order, \emph{RMSE} = .14. Standardized residuals were calculated by
dividing the difference in covariance tables by the variance of the
differences (Hausman, 1978). While \emph{RMSE} indicated partial misfit
between the covariance relationships, only 3 values were significantly
different using \emph{Z} of 1.96 as a criterion: the variances of PIL 7
and 14. PIL 7 in a randomized form had less variance (\(SD^2\) = 1.46)
than the nonrandomized form (\(SD^2\) = 1.89). Likewise, PIL 14
randomized had a smaller variance (\(SD^2\) = 1.91) than the
nonrandomized form (\(SD^2\) = 2.39). Questions about retirement and
freedom to make choices decreased in variance when they were randomly
presented.

\subsubsection{Factor Loadings}\label{factor-loadings}

Table \ref{tab:Ptable} includes the factor loadings from the one-factor
EFA analysis. These loadings were compared using a dependent
\emph{t}-test matched on item, and they were not significantly
different, \(M_d = 0.00\), 95\% CI \([-0.02\), \(0.03]\),
\(t(19) = 0.31\), \(p = .763\). The effect size for this test was
correspondingly negligible, \(d_{av}\) = -0.02 95\% CI {[}-0.46,
0.42{]}. The TOST test was significant for both the lower, \emph{t}(19)
= 0.14, \emph{p} \textless{} .001 and the upper bound, \emph{t}(19) =
-0.75, \emph{p} \textless{} .001. This result indicated that the change
score was within the confidence band of expected negligible changes.
Lastly, the BF for this test was 0.24 ±0.02\%, which indicated support
for the null model.

\subsubsection{Item Means}\label{item-means}

Table \ref{tab:Ptable} includes the means and standard deviation of each
item from the PIL scale. The item means were compared using a dependent
\emph{t}-test matched on item. Item means were significantly different
\(M_d = -0.07\), 95\% CI \([-0.13\), \(-0.02]\), \(t(19) = -2.92\),
\(p = .009\). The effect size for this difference was small, \(d_{av}\)
= -0.16 95\% CI {[}-0.60, 0.29{]}. Even though the \emph{t}-test was
significant, the TOST test indicated that the difference was within the
range of a 5\% percent change in item means (0.30). The TOST test for
lower bound, \emph{t}(19) = -1.58, \emph{p} \textless{} .001 and the
upper bound, \emph{t}(19) = -4.26, \emph{p} \textless{} .001, suggested
that the significant \emph{t}-test may be not be interpreted as a
meaningful change on the item means. The BF value for this test
indicated 6.86 \textless{}0.01\%, which is often considered weak
evidence for the alternative model. Here, we find mixed results,
indicating that randomization may change item means for the PIL.

\subsubsection{Total Scores}\label{total-scores}

\begin{table}[tbp]
\begin{center}
\begin{threeparttable}
\caption{\label{tab:Ptable}Item Statistics for the PIL Scale}
\begin{tabular}{lcccccc}
\toprule
Item & \multicolumn{1}{c}{FL-R} & \multicolumn{1}{c}{FL-NR} & \multicolumn{1}{c}{M-R} & \multicolumn{1}{c}{SD-R} & \multicolumn{1}{c}{M-NR} & \multicolumn{1}{c}{SD-NR}\\
\midrule
1 & .671 & .638 & 4.824 & 1.279 & 4.806 & 1.278\\
2 & .678 & .572 & 4.930 & 1.436 & 4.599 & 1.453\\
3 & .685 & .674 & 5.812 & 1.126 & 5.729 & 1.100\\
4 & .840 & .848 & 5.673 & 1.303 & 5.658 & 1.279\\
5 & .637 & .572 & 4.669 & 1.495 & 4.413 & 1.495\\
6 & .674 & .686 & 5.422 & 1.313 & 5.336 & 1.402\\
7 & .422 & .441 & 6.174 & 1.207 & 6.077 & 1.375\\
8 & .627 & .600 & 5.013 & 1.092 & 5.004 & 1.146\\
9 & .823 & .797 & 5.354 & 1.176 & 5.326 & 1.199\\
10 & .722 & .764 & 5.204 & 1.497 & 5.155 & 1.546\\
11 & .774 & .794 & 5.229 & 1.625 & 5.170 & 1.615\\
12 & .602 & .647 & 4.491 & 1.572 & 4.527 & 1.595\\
13 & .429 & .404 & 5.746 & 1.244 & 5.737 & 1.215\\
14 & .451 & .421 & 5.428 & 1.381 & 5.237 & 1.547\\
15 & .081 & .202 & 4.375 & 1.940 & 4.143 & 1.881\\
16 & .549 & .558 & 5.094 & 1.987 & 5.261 & 1.862\\
17 & .722 & .736 & 5.418 & 1.396 & 5.397 & 1.401\\
18 & .479 & .504 & 5.382 & 1.478 & 5.296 & 1.600\\
19 & .678 & .721 & 4.880 & 1.412 & 4.903 & 1.456\\
20 & .780 & .809 & 5.341 & 1.316 & 5.213 & 1.287\\
\bottomrule
\addlinespace
\end{tabular}
\begin{tablenotes}[para]
\textit{Note.} FL = Factor Loadings, M = Mean, SD = Standard Deviation, R = Random, NR = Not Random
\end{tablenotes}
\end{threeparttable}
\end{center}
\end{table}

Total scores were created by summing the items for each participant
across all twenty PIL questions. The mean total score for nonrandomized
testing was \emph{M} = 102.99 (\emph{SD} = 18.28), while the mean for
randomizing testing was \emph{M} = 104.46 (\emph{SD} = 17.82). This
difference was examined with an independent \emph{t}-test and was not
significant, \(t(1,897) = -1.77\), \(p = .077\). The effect size for
this difference was negligible, \(d_{av}\) = -0.08 95\% CI {[}-0.17,
0.29{]}. We tested if scores were changed by 5\% (6.00 points), and the
TOST test indicated that the lower, \emph{t}(1897) = 5.43, \emph{p}
\textless{} .001 and the upper bound, \emph{t}(1897) = -8.96, \emph{p}
\textless{} .001 were within this area of null change. The BF results
also supported the null model, 0.25 \textless{}0.01\%.

\subsection{LPQ Analyses}\label{lpq-analyses}

\subsubsection{Covariance Matrices}\label{covariance-matrices-1}

Covariance structure for the LPQ was found to be the same across both
randomized and nonrandomized testing, \emph{RMSE} = .02. Standardized
residuals indicated that the covariance between items 9 and 11 were
significantly different, while item 13 included significantly different
variances. The correlation between items 9 (empty life) and 11
(wondering about being alive) for randomized versions was \emph{r} = .33
while the correlation for nonrandomized versions was \emph{r} = .51. The
variance for item 13 (responsibility) in a randomized version (\(SD^2\)
= .03) was smaller than the variance in the nonrandomized version
(\(SD^2\) = .08).

\subsubsection{Factor Loadings}\label{factor-loadings-1}

Table \ref{tab:Ltable} includes the factor loadings from the one-factor
EFA analysis using tetrachoric correlations. The loadings from
randomized and nonrandomized versions were compared using a dependent
\emph{t}-test matched on item, which indicated they were not
significantly different, \(M_d = 0.01\), 95\% CI \([-0.02\), \(0.04]\),
\(t(19) = 0.94\), \(p = .361\). The difference found for this test was
negligible, \(d_{av}\) = -0.06 95\% CI {[}-0.50, 0.38{]}. The TOST test
examined if any change was within .10 change, as described earlier. The
lower, \emph{t}(19) = -0.49, \emph{p} \textless{} .001 and the upper
bound, \emph{t}(19) = -1.38, \emph{p} \textless{} .001 were both
significant, indicating that the change was within the expected change.
Further, in support of the null model, the BF was 0.34 ±0.02\%.

\subsubsection{Item Means}\label{item-means-1}

Means and standard deviations of each item are presented in Table
\ref{tab:Ltable}. We again matched items and tested if there was a
significant change using a dependent \emph{t}-test. The test was not
significant, \(M_d = 0.00\), 95\% CI \([-0.02\), \(0.02]\),
\(t(19) = 0.22\), \(p = .827\), and the corresponding effect size
reflects how little these means changed, \(d_{av}\) = 0.01 95\% CI
{[}-0.43, 0.45{]}. Using a 5\% change criterion, items were tested to
determine if they changed less than (0.05). The TOST test indicated both
lower, \emph{t}(19) = 0.45, \emph{p} \textless{} .001 and the upper
bound, \emph{t}(19) = 0.00, \emph{p} \textless{} .001, were within the
null range. The BF also supported the null model, 0.24 ±0.02\%.

\subsubsection{Total Scores}\label{total-scores-1}

\begin{table}[tbp]
\begin{center}
\begin{threeparttable}
\caption{\label{tab:Ltable}Item Statistics for the LPQ Scale}
\begin{tabular}{lcccccc}
\toprule
Item & \multicolumn{1}{c}{FL-R} & \multicolumn{1}{c}{FL-NR} & \multicolumn{1}{c}{M-R} & \multicolumn{1}{c}{SD-R} & \multicolumn{1}{c}{M-NR} & \multicolumn{1}{c}{SD-NR}\\
\midrule
1 & .676 & .682 & .567 & .496 & .612 & .488\\
2 & .901 & .872 & .755 & .431 & .760 & .427\\
3 & .503 & .394 & .864 & .343 & .844 & .363\\
4 & .725 & .686 & .907 & .290 & .868 & .339\\
5 & .689 & .686 & .419 & .494 & .509 & .500\\
6 & .511 & .558 & .637 & .481 & .581 & .494\\
7 & .189 & .293 & .775 & .418 & .809 & .393\\
8 & .557 & .473 & .483 & .500 & .467 & .499\\
9 & .856 & .911 & .812 & .391 & .783 & .413\\
10 & .594 & .621 & .636 & .481 & .647 & .478\\
11 & .639 & .756 & .727 & .446 & .760 & .427\\
12 & .683 & .761 & .786 & .410 & .751 & .433\\
13 & .314 & .395 & .965 & .184 & .909 & .287\\
14 & .484 & .483 & .761 & .427 & .769 & .422\\
15 & .050 & .099 & .322 & .468 & .395 & .489\\
16 & .697 & .699 & .862 & .345 & .871 & .336\\
17 & .517 & .496 & .848 & .359 & .812 & .391\\
18 & .559 & .509 & .829 & .377 & .827 & .379\\
19 & .674 & .716 & .464 & .499 & .497 & .500\\
20 & .636 & .614 & .723 & .448 & .712 & .453\\
\bottomrule
\addlinespace
\end{tabular}
\begin{tablenotes}[para]
\textit{Note.} FL = Factor Loadings, M = Mean, SD = Standard Deviation, R = Random, NR = Not Random
\end{tablenotes}
\end{threeparttable}
\end{center}
\end{table}

LPQ total scores were created by summing the items for each participant.
The mean total score for randomized testing was \emph{M} = 14.14
(\emph{SD} = 4.01), and the mean for nonrandomized testing was \emph{M}
= 14.18 (\emph{SD} = 4.22). An independent \emph{t}-test indicated that
the testing did not change total score, \(t(1,632) = 0.20\),
\(p = .844\). The effect size for this difference was negligible,
\(d_{av}\) = 0.01 95\% CI {[}-0.09, 0.45{]}. The TOST test indicated
that the scores were withing a 5\% (1.00 points) change, lower:
\emph{t}(1627) = 5.10, \emph{p} \textless{} .001 and upper:
\emph{t}(1627) = -4.71, \emph{p} \textless{} .001. The BF results were
in support of the null model, 0.06 ±0.04\%.

\section{Discussion}\label{discussion}

As technology has advanced, initial research questioned the validity of
online assessments versus paper assessments. With further investigation,
several researchers discovered measurement invariance with regard to
computer surveys compared with paper surveys (Deutskens et al., 2006;
Lewis et al., 2009). However, with the addition of technology, Fang et
al. (2012a) suggested that individuals respond with more extreme scores
in online surveys than in-person surveys due to the social-desirability
effect. Research on scale invariance is mixed in results for paper and
computer, and our work is a first-step on examining survey equivalence
on an individual item-level for different forms of computer delivery.

The findings from the current study imply that item randomization is a
viable option for controlling any potential reactivity between
questions. First, as we analyzed the PIL, the covariance matrices were
non-equivalent; the randomized data show decreased variance for several
items compared to the nonrandomized data. Since variance provides a
measure of how the data vary around the mean, decreased variance
typically results in decreased measurement error; thus, randomization
has the potential to decrease measurement error in data collection. The
findings also support the null hypothesis in regards to factor loading
differences because the item relationship to a latent variable should
not change with randomization. The item means comparison resulted in
significant differences between item randomization and nonrandomization
using \emph{p}-value criterion and Bayes Factor analyses. However, the
effect size was small, meaning the differences were not as meaningful as
the \emph{p}-values and \emph{BF} analyses posit, in addition to
considering the evidentiary values of the two one-sided tests, which
supported the null range of expected values. Finally, the total scores
showed equivalence between randomization and nonrandomization which
suggested that total scales were not considerably impacted with or
without randomization of items.

Analyses for the LPQ yielded somewhat similar results to those of the
PIL. Pertaining to covariance structures, the randomized and
nonrandomized scales resulted in equivalence, with a recapitulation of
the PIL analysis in which variance was decreased in the randomized
sample for at least one item. A slight correlational difference was
detected for items 9 and 11 in which the nonrandomized scale shows a
large association between the items, while the randomized scale shows a
moderate association between the items. However, the presence of the
association remained present on both randomized and nonrandomized
scales. Further analyses of the factor loadings, item means, and total
scores resulted in equivalence between forms. Therefore, the null
hypothesis was supported. Evidentiary equivalence for item means and
total scores suggested that randomization of items was not
disadvantaging the overall scoring structure of the scale and provides
further support for randomization as a means of methodological control.
The match between results for two types of answer methodologies
(i.e.~Likert and True/False) implied that randomization can be applied
across a variety of scale types with similar effects.

Since the PIL and LPQ analyses predominately illustrated support for
null effects of randomization, item randomization of scales is of
practical use when there are potential concerns about item order.
Randomization has been largely viewed as virtuous research practice in
terms of sample selection and order of stimuli presentation for years;
now, we must decide if item reactivity earns the same amount of caution
that has been granted to existing research procedures. Since we found
equivalence in terms of overall scoring of the PIL and LPQ, we advise
that randomization should and can be used as a control mechanism, in
addition to the ease of comparison between the scales if one researcher
decided to randomize and one did not. Moreover, these results would
imply that if an individual's total score on the PIL or LPQ is
significantly different on randomized versus nonrandomized
administrations, it is likely due to factors unrelated to delivery.
Future research should investigate if this result is WEIRD (Western,
Educated, Industrialized, Rich, and Democratic), as this study focused
on college-age students in the Midwest (Henrich, Heine, \& Norenzayan,
2010). As Fang et al. (2012b)'s research indicates different effects for
collectivistic cultures, other cultures may show different results based
on randomization. Additionally, one should consider the effects of
potential computer illiteracy on online surveys (Charters, 2004).

A second benefit to using the procedures outlined in this paper to
examine for differences in methodology is the simple implementation of
the analyses. While our analyses were performed in \emph{R}, nearly all
of these analyses can be performed in free point and click software,
such as \emph{jamovi} and \emph{JASP}. Multigroup confirmatory factory
analyses can additionally be used to analyze a very similar set of
questions (covariance matrices, latent loadings, item means, and latent
means; Brown, 2006); however, multigroup analyses require a specialized
skill and knowledge set. Bayes Factor and TOST analyses are included in
these free programs and are easy to implement. In this paper, we have
provided examples of how to test the null hypothesis, as well as ways to
include multiple forms of evidentiary value to critically judge an
analysis on facets other than \emph{p}-values (Valentine et al., 2017).

\newpage

\section{References}\label{references}

\setlength{\parindent}{-0.5in} \setlength{\leftskip}{0.5in}

\hypertarget{refs}{}
\hypertarget{ref-Benjamin2017}{}
Benjamin, D. J., Berger, J. O., Johannesson, M., Nosek, B. A.,
Wagenmakers, E.-J., Berk, R., \ldots{} Johnson, V. E. (2018). Redefine
statistical significance. \emph{Nature Human Behaviour}, \emph{2}(1),
6--10.
doi:\href{https://doi.org/10.1038/s41562-017-0189-z}{10.1038/s41562-017-0189-z}

\hypertarget{ref-Bethlehem2010}{}
Bethlehem, J. (2010). Selection bias in web surveys. \emph{International
Statistical Review}, \emph{78}(2), 161--188.
doi:\href{https://doi.org/10.1111/j.1751-5823.2010.00112.x}{10.1111/j.1751-5823.2010.00112.x}

\hypertarget{ref-Brown2006}{}
Brown, T. (2006). \emph{Confirmatory factor analysis for applied
research}. New York, NY: The Guilford Press.

\hypertarget{ref-Buchanan2014}{}
Buchanan, E. M., Valentine, K. D., \& Schulenberg, S. E. (2014).
Exploratory and Confirmatory Factor Analysis: Developing the Purpose in
Life Test--Short Form. In P. Bindle (Ed.), \emph{SAGE research methods
cases}. London, United Kingdom: SAGE Publications, Ltd.
doi:\href{https://doi.org/10.4135/978144627305013517794}{10.4135/978144627305013517794}

\hypertarget{ref-Buchanan2017}{}
Buchanan, E. M., Valentine, K. D., \& Scofield, J. E. (2017). MOTE.
Retrieved from \url{https://github.com/doomlab/MOTE}

\hypertarget{ref-Buchanan2005}{}
Buchanan, T., Ali, T., Heffernan, T., Ling, J., Parrott, A., Rodgers,
J., \& Scholey, A. (2005). Nonequivalence of on-line and
paper-and-pencil psychological tests: The case of the prospective memory
questionnaire. \emph{Behavior Research Methods}, \emph{37}(1), 148--154.
doi:\href{https://doi.org/10.3758/BF03206409}{10.3758/BF03206409}

\hypertarget{ref-Buhrmester2011}{}
Buhrmester, M., Kwang, T., \& Gosling, S. D. (2011). Amazon's Mechanical
Turk: A new source of inexpensive, yet high-quality, data?
\emph{Perspectives on Psychological Science}, \emph{6}(1), 3--5.
doi:\href{https://doi.org/10.1177/1745691610393980}{10.1177/1745691610393980}

\hypertarget{ref-Cantrell2007}{}
Cantrell, M. A., \& Lupinacci, P. (2007). Methodological issues in
online data collection. \emph{Journal of Advanced Nursing},
\emph{60}(5), 544--549.
doi:\href{https://doi.org/10.1111/j.1365-2648.2007.04448.x}{10.1111/j.1365-2648.2007.04448.x}

\hypertarget{ref-Charters2004}{}
Charters, E. (2004). New perspectives on popular culture, science and
technology: Web browsers and the new illiteracy. \emph{College
Quarterly}, \emph{7}(1), 1--13.

\hypertarget{ref-Cohen1992a}{}
Cohen, J. (1992). A power primer. \emph{Psychological Bulletin},
\emph{112}(1), 155--159.
doi:\href{https://doi.org/10.1037//0033-2909.112.1.155}{10.1037//0033-2909.112.1.155}

\hypertarget{ref-Cook2000}{}
Cook, C., Heath, F., \& Thompson, R. L. (2000). A meta-analysis of
response rates in Web- or Internet-based surveys. \emph{Educational and
Psychological Measurement}, \emph{60}(6), 821--836.
doi:\href{https://doi.org/10.1177/00131640021970934}{10.1177/00131640021970934}

\hypertarget{ref-Cribbie2004}{}
Cribbie, R. A., Gruman, J. A., \& Arpin-Cribbie, C. A. (2004).
Recommendations for applying tests of equivalence. \emph{Journal of
Clinical Psychology}, \emph{60}(1), 1--10.
doi:\href{https://doi.org/10.1002/jclp.10217}{10.1002/jclp.10217}

\hypertarget{ref-Cronk2002}{}
Cronk, B. C., \& West, J. L. (2002). Personality research on the
Internet: A comparison of Web-based and traditional instruments in
take-home and in-class settings. \emph{Behavior Research Methods,
Instruments, \& Computers}, \emph{34}(2), 177--180.
doi:\href{https://doi.org/10.3758/BF03195440}{10.3758/BF03195440}

\hypertarget{ref-Crumbaugh1964}{}
Crumbaugh, J. C., \& Maholick, L. T. (1964). An experimental study in
existentialism: The psychometric approach to Frankl's concept ofnoogenic
neurosis. \emph{Journal of Clinical Psychology}, \emph{20}(2), 200--207.
doi:\href{https://doi.org/10.1002/1097-4679(196404)20:2\%3C200::AID-JCLP2270200203\%3E3.0.CO;2-U}{10.1002/1097-4679(196404)20:2\textless{}200::AID-JCLP2270200203\textgreater{}3.0.CO;2-U}

\hypertarget{ref-Cumming2014}{}
Cumming, G. (2014). The new statistics: Why and how. \emph{Psychological
Science}, \emph{25}(1), 7--29.
doi:\href{https://doi.org/10.1177/0956797613504966}{10.1177/0956797613504966}

\hypertarget{ref-DeLeeuw1988}{}
De Leeuw, E. D., \& Hox, J. J. (1988). The effects of
response-stimulating factors on response rates and data quality in mail
surveys: A test of Dillman's total design method. \emph{Journal of
Official Statistics}, \emph{4}(3), 241--249.

\hypertarget{ref-Deutskens2006}{}
Deutskens, E., Ruyter, K. de, \& Wetzels, M. (2006). An Assessment of
Equivalence Between Online and Mail Surveys in Service Research.
\emph{Journal of Service Research}, \emph{8}(4), 346--355.
doi:\href{https://doi.org/10.1177/1094670506286323}{10.1177/1094670506286323}

\hypertarget{ref-DeVellis2016a}{}
DeVellis, R. F. (2016). \emph{Scale Development: Theory and
Applications, 4th Edition} (Vol. 26).

\hypertarget{ref-Dillman2008}{}
Dillman, D. A., Smyth, J. D., \& Christian, L. M. (2008).
\emph{Internet, mail, and mixed-mode surveys: The tailored design
method} (3rd ed.). Hoboken, NJ: John Wiley \& Sons, Inc.
doi:\href{https://doi.org/10.2307/41061275}{10.2307/41061275}

\hypertarget{ref-Fang2012a}{}
Fang, J., Wen, C., \& Pavur, R. (2012a). Participation willingness in
web surveys: Exploring effect of sponsoring corporation's and survey
provider's reputation. \emph{Cyberpsychology, Behavior, and Social
Networking}, \emph{15}(4), 195--199.
doi:\href{https://doi.org/10.1089/cyber.2011.0411}{10.1089/cyber.2011.0411}

\hypertarget{ref-Fang2012}{}
Fang, J., Wen, C., \& Prybutok, V. R. (2012b). An assessment of
equivalence between Internet and paper-based surveys: evidence from
collectivistic cultures. \emph{Quality \& Quantity}, \emph{48}(1),
493--506.
doi:\href{https://doi.org/10.1007/s11135-012-9783-3}{10.1007/s11135-012-9783-3}

\hypertarget{ref-Frick2001}{}
Frick, A., Bächtiger, M. T., \& Reips, U.-D. (2001). Financial
incentives, personal information and dropout in online studies. In U.-D.
Reips \& M. Bosnjak (Eds.), \emph{Dimensions of internet science} (pp.
209--219).

\hypertarget{ref-Gramacy2010}{}
Gramacy, R. B., \& Pantaleoy, E. (2010). Shrinkage regression for
multivariate inference with missing data, and an application to
portfolio balancing. \emph{Bayesian Analysis}, \emph{5}(2), 237--262.
doi:\href{https://doi.org/10.1214/10-BA602}{10.1214/10-BA602}

\hypertarget{ref-Hausman1978}{}
Hausman, J. A. (1978). Specification tests in Econometrics.
\emph{Econometrica}, \emph{46}(6), 1251.
doi:\href{https://doi.org/10.2307/1913827}{10.2307/1913827}

\hypertarget{ref-Henrich2010}{}
Henrich, J., Heine, S. J., \& Norenzayan, A. (2010). The weirdest people
in the world? \emph{Behavioral and Brain Sciences}, \emph{33}(2-3),
61--83.
doi:\href{https://doi.org/10.1017/S0140525X0999152X}{10.1017/S0140525X0999152X}

\hypertarget{ref-Hox1994}{}
Hox, J. J., \& De Leeuw, E. D. (1994). A comparison of nonresponse in
mail, telephone, and face-to-face surveys. \emph{Quality and Quantity},
\emph{28}(4), 329--344.
doi:\href{https://doi.org/10.1007/BF01097014}{10.1007/BF01097014}

\hypertarget{ref-Hu1999}{}
Hu, L., \& Bentler, P. M. (1999). Cutoff criteria for fit indexes in
covariance structure analysis: Conventional criteria versus new
alternatives. \emph{Structural Equation Modeling: A Multidisciplinary
Journal}, \emph{6}(1), 1--55.
doi:\href{https://doi.org/10.1080/10705519909540118}{10.1080/10705519909540118}

\hypertarget{ref-Hutzell1988}{}
Hutzell, R. (1988). A review of the Purpose in Life Test.
\emph{International Forum for Logotherapy}, \emph{11}(2), 89--101.

\hypertarget{ref-Ilieva2001}{}
Ilieva, J., Baron, S., \& Healy, N. M. (2002). On-line surveys in
international marketing research: Pros and cons. \emph{International
Journal of Market Research}, \emph{44}(3), 361--376.

\hypertarget{ref-Joinson1999}{}
Joinson, A. (1999). Social desirability, anonymity, and Intemet-based
questionnaires. \emph{Behavior Research Methods, Instruments, \&
Computers}, \emph{31}(3), 433--438.
doi:\href{https://doi.org/10.3758/BF03200723}{10.3758/BF03200723}

\hypertarget{ref-Keppel2004}{}
Keppel, G., \& Wickens, T. (2004). \emph{Design and Analysis: A
Researcher's Handbook} (4th ed.). Upper Saddle River, NJ: Prentice Hall.

\hypertarget{ref-Lakens2013}{}
Lakens, D. (2013). Calculating and reporting effect sizes to facilitate
cumulative science: A practical primer for t-tests and ANOVAs.
\emph{Frontiers in Psychology}, \emph{4}.
doi:\href{https://doi.org/10.3389/fpsyg.2013.00863}{10.3389/fpsyg.2013.00863}

\hypertarget{ref-Lakens2017a}{}
Lakens, D. (2017). Equivalence tests. \emph{Social Psychological and
Personality Science}, \emph{8}(4), 355--362.
doi:\href{https://doi.org/10.1177/1948550617697177}{10.1177/1948550617697177}

\hypertarget{ref-Lakens2017}{}
Lakens, D., Adolfi, F. G., Albers, C. J., Anvari, F., Apps, M. A. J.,
Argamon, S. E., \ldots{} Zwaan, R. A. (n.d.). Justify Your Alpha.
\emph{Nature Human Behaviour}.
doi:\href{https://doi.org/10.17605/OSF.IO/9S3Y6}{10.17605/OSF.IO/9S3Y6}

\hypertarget{ref-Lewis2009}{}
Lewis, I., Watson, B., \& White, K. M. (2009). Internet versus
paper-and-pencil survey methods in psychological experiments:
Equivalence testing of participant responses to health-related messages.
\emph{Australian Journal of Psychology}, \emph{61}(2), 107--116.
doi:\href{https://doi.org/10.1080/00049530802105865}{10.1080/00049530802105865}

\hypertarget{ref-Media2016}{}
Media. (2016). \emph{The Total Audience Report: Q1 2016}.

\hypertarget{ref-Melton2008}{}
Melton, A. M. A., \& Schulenberg, S. E. (2008). On the measurement of
meaning: Logotherapy's empirical contributions to humanistic psychology.
\emph{The Humanistic Psychologist}, \emph{36}(1), 31--44.
doi:\href{https://doi.org/10.1080/08873260701828870}{10.1080/08873260701828870}

\hypertarget{ref-Meyerson2003}{}
Meyerson, P., \& Tryon, W. W. (2003). Validating Internet research: A
test of the psychometric equivalence of Internet and in-person samples.
\emph{Behavior Research Methods, Instruments, \& Computers},
\emph{35}(4), 614--620.
doi:\href{https://doi.org/10.3758/BF03195541}{10.3758/BF03195541}

\hypertarget{ref-Morey2015c}{}
Morey, R. D. (2015). On verbal categories for the interpretation of
Bayes factors. Retrieved from
\url{http://bayesfactor.blogspot.com/2015/01/on-verbal-categories-for-interpretation.html}

\hypertarget{ref-Morey2015b}{}
Morey, R. D., \& Rouder, J. N. (2015). BayesFactor: Computation of Bayes
Factors for common designs. Retrieved from
\url{https://cran.r-project.org/package=BayesFactor}

\hypertarget{ref-Musch2000}{}
Musch, J., \& Reips, U.-D. (2000). A brief history of web experimenting.
In M. H. Birnbaum (Ed.), \emph{Psychological experiments on the
internet} (pp. 61--87). Elsevier.
doi:\href{https://doi.org/10.1016/B978-012099980-4/50004-6}{10.1016/B978-012099980-4/50004-6}

\hypertarget{ref-Nosek2002}{}
Nosek, B. A., Banaji, M. R., \& Greenwald, A. G. (2002). E-Research:
Ethics, security, design, and control in psychological research on the
Internet. \emph{Journal of Social Issues}, \emph{58}(1), 161--176.
doi:\href{https://doi.org/10.1111/1540-4560.00254}{10.1111/1540-4560.00254}

\hypertarget{ref-Olson2010}{}
Olson, K. (2010). An examination of questionnaire evaluation by expert
reviewers. \emph{Field Methods}, \emph{22}(4), 295--318.
doi:\href{https://doi.org/10.1177/1525822X10379795}{10.1177/1525822X10379795}

\hypertarget{ref-Preacher2003}{}
Preacher, K. J., \& MacCallum, R. C. (2003). Repairing Tom Swift's
Electric Factor Analysis Machine. \emph{Understanding Statistics},
\emph{2}(1), 13--43.
doi:\href{https://doi.org/10.1207/S15328031US0201_02}{10.1207/S15328031US0201\_02}

\hypertarget{ref-Reips2002a}{}
Reips, U.-D. (2002). Standards for Internet-based experimenting.
\emph{Experimental Psychology}, \emph{49}(4), 243--256.
doi:\href{https://doi.org/10.1026//1618-3169.49.4.243}{10.1026//1618-3169.49.4.243}

\hypertarget{ref-Reips2012}{}
Reips, U.-D. (2012). Using the Internet to collect data. In \emph{APA
handbook of research methods in psychology, vol 2: Research designs:
Quantitative, qualitative, neuropsychological, and biological.} (Vol. 2,
pp. 291--310). Washington: American Psychological Association.
doi:\href{https://doi.org/10.1037/13620-017}{10.1037/13620-017}

\hypertarget{ref-Rogers1993}{}
Rogers, J. L., Howard, K. I., \& Vessey, J. T. (1993). Using
significance tests to evaluate equivalence between two experimental
groups. \emph{Psychological Bulletin}, \emph{113}(3), 553--565.
doi:\href{https://doi.org/10.1037/0033-2909.113.3.553}{10.1037/0033-2909.113.3.553}

\hypertarget{ref-Rouder2009}{}
Rouder, J. N., Speckman, P. L., Sun, D., Morey, R. D., \& Iverson, G.
(2009). Bayesian t tests for accepting and rejecting the null
hypothesis. \emph{Psychonomic Bulletin \& Review}, \emph{16}(2),
225--237.
doi:\href{https://doi.org/10.3758/PBR.16.2.225}{10.3758/PBR.16.2.225}

\hypertarget{ref-Sanou2017}{}
Sanou, B. (2017, July). ICT Facts and Figures 2017. Retrieved from
\url{http://www.itu.int/en/ITU-D/Statistics/Documents/facts/ICTFactsFigures2017.pdf}

\hypertarget{ref-Schuirmann1987}{}
Schuirmann, D. J. (1987). A comparison of the Two One-Sided Tests
Procedure and the Power Approach for assessing the equivalence of
average bioavailability. \emph{Journal of Pharmacokinetics and
Biopharmaceutics}, \emph{15}(6), 657--680.
doi:\href{https://doi.org/10.1007/BF01068419}{10.1007/BF01068419}

\hypertarget{ref-Schuldt1994}{}
Schuldt, B. a., \& Totten, J. W. (1994). Electronic mail vs. mail survey
response rates. \emph{Marketing Research}, \emph{6}, 36--39.

\hypertarget{ref-Schulenberg2004}{}
Schulenberg, S. E. (2004). A psychometric investigation of logotherapy
measures and the Outcome Questionnaire (OQ-45.2). \emph{North American
Journal of Psychology}, \emph{6}(3), 477--492.

\hypertarget{ref-Schulenberg2010}{}
Schulenberg, S. E., \& Melton, A. M. A. (2010). A confirmatory
factor-analytic evaluation of the purpose in life test: Preliminary
psychometric support for a replicable two-factor model. \emph{Journal of
Happiness Studies}, \emph{11}(1), 95--111.
doi:\href{https://doi.org/10.1007/s10902-008-9124-3}{10.1007/s10902-008-9124-3}

\hypertarget{ref-Schulenberg1999}{}
Schulenberg, S. E., \& Yutrzenka, B. A. (1999). The equivalence of
computerized and paper-and-pencil psychological instruments:
Implications for measures of negative affect. \emph{Behavior Research
Methods, Instruments, \& Computers}, \emph{31}(2), 315--321.
doi:\href{https://doi.org/10.3758/BF03207726}{10.3758/BF03207726}

\hypertarget{ref-Schulenberg2001}{}
Schulenberg, S. E., \& Yutrzenka, B. A. (2001). Equivalence of
computerized and conventional versions of the Beck Depression
Inventory-II (BDI-II). \emph{Current Psychology}, \emph{20}(3),
216--230.
doi:\href{https://doi.org/10.1007/s12144-001-1008-1}{10.1007/s12144-001-1008-1}

\hypertarget{ref-Schulenberg2011}{}
Schulenberg, S. E., Schnetzer, L. W., \& Buchanan, E. M. (2011). The
Purpose in Life Test-Short Form: Development and Psychometric Support.
\emph{Journal of Happiness Studies}, \emph{12}(5), 861--876.
doi:\href{https://doi.org/10.1007/s10902-010-9231-9}{10.1007/s10902-010-9231-9}

\hypertarget{ref-Smithson2001}{}
Smithson, M. (2001). Correct confidence intervals for various regression
effect sizes and parameters: The importance of noncentral distributions
in computing intervals. \emph{Educational and Psychological
Measurement}, \emph{61}(4), 605--632.
doi:\href{https://doi.org/10.1177/00131640121971392}{10.1177/00131640121971392}

\hypertarget{ref-Smyth2006}{}
Smyth, J. D. (2006). Comparing check-all and forced-choice question
formats in web surveys. \emph{Public Opinion Quarterly}, \emph{70}(1),
66--77.
doi:\href{https://doi.org/10.1093/poq/nfj007}{10.1093/poq/nfj007}

\hypertarget{ref-Strack2009}{}
Strack, K. M., \& Schulenberg, S. E. (2009). Understanding empowerment,
meaning, and perceived coercion in individuals with serious mental
illness. \emph{Journal of Clinical Psychology}, \emph{65}(10),
1137--1148.
doi:\href{https://doi.org/10.1002/jclp.20607}{10.1002/jclp.20607}

\hypertarget{ref-Tabachnick2012}{}
Tabachnick, B. G., \& Fidell, L. S. (2012). \emph{Using Multivariate
Statistics} (6th ed.). Boston, MA: Pearson.

\hypertarget{ref-Tourangeau1999}{}
Tourangeau, R., Rips, L. J., \& Rasinski, K. (1999). \emph{The
psychology of survey response}. Cambridge, UK: Cambridge University
Press.

\hypertarget{ref-Valentine2017}{}
Valentine, K. D., Buchanan, E. M., Scofield, J. E., \& Beauchamp, M.
(2017). Beyond p-values: Utilizing Multiple Estimates to Evaluate
Evidence.
doi:\href{https://doi.org/10.17605/osf.io/9hp7y}{10.17605/osf.io/9hp7y}

\hypertarget{ref-VanBuuren2011}{}
Van Buuren, S., \& Groothuis-Oudshoorn, K. (2011). mice: Multivariate
Imputation by Chained Equations in R. \emph{Journal of Statistical
Software}, \emph{45}(3), 1--67.
doi:\href{https://doi.org/10.18637/jss.v045.i03}{10.18637/jss.v045.i03}

\hypertarget{ref-Weigold2013}{}
Weigold, A., Weigold, I. K., \& Russell, E. J. (2013). Examination of
the equivalence of self-report survey-based paper-and-pencil and
internet data collection methods. \emph{Psychological Methods},
\emph{18}(1), 53--70.
doi:\href{https://doi.org/10.1037/a0031607}{10.1037/a0031607}

\hypertarget{ref-Worthington2006}{}
Worthington, R. L., \& Whittaker, T. a. (2006). Scale development
research: A content analysis and recommendations for best practices.
\emph{The Counseling Psychologist}, \emph{34}(6), 806--838.
doi:\href{https://doi.org/10.1177/0011000006288127}{10.1177/0011000006288127}






\end{document}
