\documentclass[english,man, mask]{apa6}

\usepackage{amssymb,amsmath}
\usepackage{ifxetex,ifluatex}
\usepackage{fixltx2e} % provides \textsubscript
\ifnum 0\ifxetex 1\fi\ifluatex 1\fi=0 % if pdftex
  \usepackage[T1]{fontenc}
  \usepackage[utf8]{inputenc}
\else % if luatex or xelatex
  \ifxetex
    \usepackage{mathspec}
    \usepackage{xltxtra,xunicode}
  \else
    \usepackage{fontspec}
  \fi
  \defaultfontfeatures{Mapping=tex-text,Scale=MatchLowercase}
  \newcommand{\euro}{€}
\fi
% use upquote if available, for straight quotes in verbatim environments
\IfFileExists{upquote.sty}{\usepackage{upquote}}{}
% use microtype if available
\IfFileExists{microtype.sty}{\usepackage{microtype}}{}

% Table formatting
\usepackage{longtable, booktabs}
\usepackage{lscape}
% \usepackage[counterclockwise]{rotating}   % Landscape page setup for large tables
\usepackage{multirow}		% Table styling
\usepackage{tabularx}		% Control Column width
\usepackage[flushleft]{threeparttable}	% Allows for three part tables with a specified notes section
\usepackage{threeparttablex}            % Lets threeparttable work with longtable

% Create new environments so endfloat can handle them
% \newenvironment{ltable}
%   {\begin{landscape}\begin{center}\begin{threeparttable}}
%   {\end{threeparttable}\end{center}\end{landscape}}

\newenvironment{lltable}
  {\begin{landscape}\begin{center}\begin{ThreePartTable}}
  {\end{ThreePartTable}\end{center}\end{landscape}}

  \usepackage{ifthen} % Only add declarations when endfloat package is loaded
  \ifthenelse{\equal{\string man, mask}{\string man}}{%
   \DeclareDelayedFloatFlavor{ThreePartTable}{table} % Make endfloat play with longtable
   % \DeclareDelayedFloatFlavor{ltable}{table} % Make endfloat play with lscape
   \DeclareDelayedFloatFlavor{lltable}{table} % Make endfloat play with lscape & longtable
  }{}%



% The following enables adjusting longtable caption width to table width
% Solution found at http://golatex.de/longtable-mit-caption-so-breit-wie-die-tabelle-t15767.html
\makeatletter
\newcommand\LastLTentrywidth{1em}
\newlength\longtablewidth
\setlength{\longtablewidth}{1in}
\newcommand\getlongtablewidth{%
 \begingroup
  \ifcsname LT@\roman{LT@tables}\endcsname
  \global\longtablewidth=0pt
  \renewcommand\LT@entry[2]{\global\advance\longtablewidth by ##2\relax\gdef\LastLTentrywidth{##2}}%
  \@nameuse{LT@\roman{LT@tables}}%
  \fi
\endgroup}


\ifxetex
  \usepackage[setpagesize=false, % page size defined by xetex
              unicode=false, % unicode breaks when used with xetex
              xetex]{hyperref}
\else
  \usepackage[unicode=true]{hyperref}
\fi
\hypersetup{breaklinks=true,
            pdfauthor={},
            pdftitle={Does the Delivery Matter? Examining Randomization at the Item Level},
            colorlinks=true,
            citecolor=blue,
            urlcolor=blue,
            linkcolor=black,
            pdfborder={0 0 0}}
\urlstyle{same}  % don't use monospace font for urls

\setlength{\parindent}{0pt}
%\setlength{\parskip}{0pt plus 0pt minus 0pt}

\setlength{\emergencystretch}{3em}  % prevent overfull lines

\ifxetex
  \usepackage{polyglossia}
  \setmainlanguage{}
\else
  \usepackage[english]{babel}
\fi

% Manuscript styling
\captionsetup{font=singlespacing,justification=justified}
\usepackage{csquotes}
\usepackage{upgreek}

 % Line numbering
  \usepackage{lineno}
  \linenumbers


\usepackage{tikz} % Variable definition to generate author note

% fix for \tightlist problem in pandoc 1.14
\providecommand{\tightlist}{%
  \setlength{\itemsep}{0pt}\setlength{\parskip}{0pt}}

% Essential manuscript parts
  \title{Does the Delivery Matter? Examining Randomization at the Item Level}

  \shorttitle{Item Randomization}


  \author{Erin M. Buchanan\textsuperscript{1}, Riley E. Foreman\textsuperscript{1}, Becca N. Johnson\textsuperscript{1}, Jeffrey M. Pavlacic\textsuperscript{2}, Rachel L. Swadley\textsuperscript{1}, \& Stefan E. Schulenberg\textsuperscript{2}}

  % \def\affdep{{"", "", "", "", "", ""}}%
  % \def\affcity{{"", "", "", "", "", ""}}%

  \affiliation{
    \vspace{0.5cm}
          \textsuperscript{1} Missouri State University\\
          \textsuperscript{2} University of Mississippi  }

  \authornote{
    Erin M. Buchanan is an Associate Professor of Quantitative Psychology at
    Missouri State University. Riley E. Foreman received his undergraduate
    degree in Psychology and Cell and Molecular Biology at Missouri State
    University and is currently at Kansas City University of Medicine and
    Biosciences. Becca N. Johnson is a masters degree candidate at Missouri
    State University. Jeffrey M. Pavlacic is a doctoral candidate at The
    University of Mississippi. Rachel N. Swadley completed her master's
    degree in Psychology at Missouri State University. Stefan E. Schulenberg
    is a Professor of Clinical Psychology at The University of Mississippi
    and Director of the Clinical Disaster Research Center. On behalf of all
    authors, the corresponding author states that there is no conflict of
    interest.
    
    Correspondence concerning this article should be addressed to Erin M.
    Buchanan, 901 S. National Ave. E-mail:
    \href{mailto:erinbuchanan@missouristate.edu}{\nolinkurl{erinbuchanan@missouristate.edu}}
  }


  \abstract{Scales that are psychometrically sound, meaning those that meet
established standards regarding reliability and validity when measuring
one or more constructs of interest, are customarily evaluated based on a
set modality (i.e., computer or paper) and administration (fixed-item
order). Deviating from an established administration profile could
result in non-equivalent response patterns, indicating the possible
evaluation of a dissimilar construct. Randomizing item administration
may alter or eliminate these effects. Therefore, we examined the
differences in scale relationships for randomized and nonrandomized
computer delivery for two scales measuring meaning/purpose in life.
These scales have questions about suicidality, depression, and life
goals that may cause item reactivity (i.e., a changed response to a
second item based on the answer to the first item). Results indicated
that item randomization does not alter scale psychometrics for meaning
in life scales, which implies that results are comparable even if
researchers implement different delivery modalities.}
  \keywords{scales, randomization, item analysis \\

    
  }





\usepackage{amsthm}
\newtheorem{theorem}{Theorem}
\newtheorem{lemma}{Lemma}
\theoremstyle{definition}
\newtheorem{definition}{Definition}
\newtheorem{corollary}{Corollary}
\newtheorem{proposition}{Proposition}
\theoremstyle{definition}
\newtheorem{example}{Example}
\theoremstyle{definition}
\newtheorem{exercise}{Exercise}
\theoremstyle{remark}
\newtheorem*{remark}{Remark}
\newtheorem*{solution}{Solution}
\begin{document}

\maketitle

\setcounter{secnumdepth}{0}



The use of the Internet has been integrated into daily life as a means
of accessing information, interacting with others, and tending to
required tasks. The International Telecommunication Union reports that
over half the world is online, and 70\% of 15-24 year olds are on the
internet (Sanou, 2017). Further, the Nielson Total Audience report from
2016 indicates that Americans spend nearly 11 hours a day in media
consumption (Media, 2016). Researchers discovered that online data
collection can be advantageous over laboratory and paper data
collection, as it is often cheaper and more efficient (Ilieva, Baron, \&
Healy, 2002; Reips, 2012; Schuldt \& Totten, 1994). Internet
questionnaires first appeared in the early 90s when HTML scripting code
integrated form elements, and the first experiments appeared soon after
(Musch \& Reips, 2000; Reips, 2002). The first experimental lab on the
internet was the Web Experimental Psychology Lab formed by Reips
(\url{http://www.wexlab.eu}), and the use of the Internet to collect
data has since grown rapidly (Reips, 2002). What started with email and
HTML forms has since moved to whole communities of available
participants including websites like Amazon's Mechanical Turk and
Qualtrics' Participant Panels. Participants of all types and forms are
easily accessible for somewhat little to no cost.

Our ability to collect data on the Internet has inevitably lead to the
question of equivalence between in person and online data collection
methods (Buchanan et al., 2005; Meyerson \& Tryon, 2003). We will use
the term equivalence as a global term for measurement of the same
underlying construct between groups, forms, or testing procedures given
no other manipulations. A related concept is measurement invariance,
which focuses on the statistical and psychometric structure of
measurement (Brown, 2006; Meredith, 1993). Multigroup confirmatory
factor analysis (MGCFA) and multiple-indicators-multiple causes (MIMIC)
structural models are often used to explore invariance in groups (Brown,
2006; Steenkamp \& Baumgartner, 1998). The general approach through
MGCFA explores if the latent structure of the proposed model is similar
across groups (equal form or configural invariance), followed by more
stringent tests indicating equal factor loadings (metric invariance),
equal item intercepts (scalar invariance), and potentially, equal error
variances (strict invariance). These steps can be used to determine
where and how groups differ when providing responses to questionnaires
and to propose changes to interpretations of test scores (for an
example, see Trent et al., 2013). Measurement invariance implies
equivalence between examined groups, while overall equivalence studies
may not imply the psychometric concept of invariance.

Research has primarily focused on simple equivalence, with more uptick
in research that specifically focuses on measurement invariance with the
advent of programs that make such procedures easier. When focusing on
equivalence, Deutskens, Ruyter, and Wetzels (2006) found that mail
surveys and online surveys produce nearly identical results regarding
the accuracy of the data collected online versus by mail. Only minor
differences arise between online surveys and mail in surveys when it
comes to participant honesty and suggestions. For example, participants
who responded to surveys online provided more suggestions, lengthier
answers, and greater information about competitors in the field that
they may prefer (Deutskens et al., 2006). The hypothesis as to why
individuals may be more honest online than in person is that the
individual may feel more anonymity and less social desirability effects
due to the nature of the online world, therefore less concerned about
responding in a socially polite way (Joinson, 1999). A trend found by
Fang, Wen, and Pavur (2012a) shows individuals are more likely to
respond to surveys online with extreme scores, rather than mid-range
responses on scales due to the lessened social desirability factor.
There may be slight cultural differences in responses online. For
example, collectivistic cultures showed greater tendency toward
mid-range responses on scales via in-person and online due to placing
greater value on how they are socially perceived; however, the trend is
still the same as scores are more extreme online versus in person or by
mail (Fang, Wen, \& Prybutok, 2012b).

Although work by Dillman and his group (Dillman, Smyth, \& Christian,
2008; Frick, Bächtiger, \& Reips, 2001; Smyth, 2006), among others, has
shown that many web surveys are plagued by problems of usability,
display, coverage, sampling, non-response, or technology, other studies
have found internet data to be reliable and almost preferable as it
produces a varied demographic response compared to the traditional
sample of introduction to psychology college students while also
maintaining equivalence (Lewis, Watson, \& White, 2009). However,
equivalence in factor structure may be problematic, as Buchanan et al.
(2005) have shown that factor structure was not replicable in online and
in person surveys. Other work has shown equivalence using a comparison
of correlation matrices (Meyerson \& Tryon, 2003) or \emph{t}-tests
(Schulenberg \& Yutrzenka, 1999, 2001), and the literature is mixed on
how different methodologies impact factor structure. Weigold, Weigold,
and Russell (2013) recently examined both quantitative and research
design questions (i.e., missing data) on Internet and paper-and-pencil
administration which showed that the administrations were generally
equivalent for quantitative structure but research design issues showed
non-equivalence. Other potential limitations to online surveys include
the accessibility of different populations to the Internet (Frick et
al., 2001), selection bias (Bethlehem, 2010), response rates (Cantrell
\& Lupinacci, 2007; Cook, Heath, \& Thompson, 2000; De Leeuw \& Hox,
1988; Hox \& De Leeuw, 1994), attrition (Cronk \& West, 2002), and
distraction (Tourangeau, Rips, \& Rasinski, 1999). Many of these
concerns have been alleviated in the years since online surveys were
first developed, especially with the advent of panels and Mechanical
Turk to reach a large, diverse population of participants (Buhrmester,
Kwang, \& Gosling, 2011).

With the development of advanced online survey platforms such as
Qualtrics and Survey Monkey, researchers have the potential to control
for confounding research design issues through randomization, although
other issues may still be present, such as participant misbehavior
(Nosek, Banaji, \& Greenwald, 2002). Randomization has been a hallmark
of good research practice, as the order or presentation of stimuli can
be a noise variable in a study with multiple measures (Keppel \&
Wickens, 2004). Thus, researchers have often randomized scales by
rotating the order of presentation in paper format or simply clicking
the randomization button for web-based studies. This practice has
counterbalanced out any order effects of going from one scale to the
next (Keppel \& Wickens, 2004). However, while scale structure has
remained constant, these items are still stimuli within a larger
construct. Therefore, these construct-related items have the ability to
influence the items that appear later on the survey, which we call item
reactivity. For example, a question about being \emph{prepared for
death} or \emph{thoughts about suicide} might change the responses to
further questions, especially if previous questions did not alert
participants to be prepared for that subject matter.

Scale development typically starts with an underlying latent variable
that a researcher wishes to examine through measured items or questions
(DeVellis, 2016). Question design is a well-studied area that indicates
that measurement is best achieved through questions that are direct,
positively worded, and understandable to the subject (Dillman et al.,
2008). Olson (2010) suggests researchers design a multitude of items in
order to investigate and invite subject matter experts to examine these
questions. Subject matter experts were found to be variable in their
agreement, but excellent at identifying potentially problematic
questions. After suggested edits from these experts, a large sample of
participant data is collected. While item response theory is gaining
traction, classical test theory has dominated this area through the use
of exploratory and confirmatory factor analysis (EFA, CFA; Worthington
\& Whittaker, 2006). EFA elucidates several facets of how the measured
items represent the latent trait through factor loadings (Tabachnick \&
Fidell, 2012). Factor structure represents the correlation between item
scores and factors, where a researcher wishes to find items that are
strongly related to latent traits. Items that are not related to the
latent trait, usually with factor loadings below .300 (Preacher \&
MacCallum, 2003) are discarded. Model fit is examined when simple
structure has been achieved (i.e., appropriate factor loadings for each
item), and these fit indices inform if the items and factor structure
model fit the data well. Well-designed scales include items that are
highly related to their latent trait and have excellent fit indices.
Scale development additionally includes the examination of other
measures of reliability and validity but the focus of the scale shifts
to subscale or total scores (Buchanan, Valentine, \& Schulenberg, 2014).
Published scales are then distributed for use in the form that is
presented in the publication, as item order is often emphasized through
important notes about reverse scoring and creating subscale scores.

The question is no longer whether web-based surveys are reliable sources
of data collection; the theory now is in need of a shift to whether or
not item-randomization in survey data collection creates psychometric
differences. These scale development procedures focus on items, and
EFA/CFA statistically try to mimic variance-covariance structure by
creating models of the data with the same variance-covariance matrix. If
we imagine that stimuli in a classic experimental design can influence
the outcome of a study because of their order, then certainly the
stimuli on a scale (i.e., the items) can influence the pattern of
responses for items. Measuring an attitude or phenomena invokes a
reaction in the participant (Knowles et al., 1992). Often, this reaction
or reactivity is treated as error in measurement, rather than a variable
to be considered in the experiment (Webb, Campbell, Schwartz, \&
Sechrest, 1966). Potentially, reaction to items on a survey could
integrate self-presentation or social desirability (Webb et al., 1966)
but cognitive factors also contribute to the participant response.
Rogers (1974) and Tourangeau and Rasinski (1988) suggested a four part
integration process that occurs when responses are formulated to
questions. First, the participant must interpret the item. The
interpretation process usually allows for one construal, and other
interpretations may be ignored (Lord, Lepper, \& Preston, 1984). Based
on this process, information about the item must be pulled from memory.
The availability heuristic will bias information found for the next
stage, the judgment process, especially given the mood of the
participant (MacLeod \& Campbell, 1992; Tversky \& Kahneman, 1973).
These memories and information, by being recalled as part of answering
an item, are often strengthened for future judgments or recall (Bargh \&
Pratto, 1986; Posner, 1978).

The judgment process has important consequences for the answers provided
on a questionnaire. Judgments are often polarized because of the
cognitive processes used to provide that answer (Tesser, 1978). The
participant may become more committed to the answer provided (Feldman \&
Lynch, 1988), and future judgments are \enquote{anchored} against this
initial judgment (Higgins \& Lurie, 1983; Strack, Schwarz, \&
Gschneidinger, 1985). Finally, future memory searches will be
confirmatory for the judgment decision (Petty \& Cacioppo, 1986). The
response selection is the final stage of the Rogers (1974) and
Tourangeau and Rasinski (1988) models. This model provides an excellent
framework through which to view the consequences of merely being asked a
question. In this study, the focus is on the final stage of response
selection, as it is the recordable output of these cognitive processes.
Knowles et al. (1992) discuss that the item order may create a context
effect for each subsequent question, wherein participants are likely to
confuse the content of an item with the context of the previous
questions. Their meaning-change hypothesis posits that each following
item will be influenced by the previous set of items and does have
important consequences for the factor loadings and reliability of the
scale. Indeed, Salancik and Brand (1992) indicate that item order
creates a specific context that integrates with background knowledge
during the answering process, which can create ambiguity in measurement
of the interested phenomenon. Panter, Tanaka, and Wellens (1992) discuss
these effects from classic studies of item ordering, wherein agreement
to a specific item first reduces agreement to a more general item second
(Strack \& Martin, 1987).

Given this previous research on item orderings, this study focuses on
potential differences in results based on item randomization delivery
methodology. This work is especially timely given the relative ease with
which randomization can be induced with survey software. The current
project examined large samples on two logotherapy-related scales, as
these scales include potentially reactive items (e.g., death and suicide
items embedded in positive psychology questions), as well as both a
dichotomous True/False and traditional 1-7 format for the same items.
Large samples were desirable to converge on a stable, representative
population; however, false positives (i.e., Type I errors) can occur by
using large \emph{N}. Recent developments in the literature focusing on
null hypothesis testing make it especially important to present
potential alternatives to \emph{p}-values (Valentine, Buchanan,
Scofield, \& Beauchamp, 2017). While a large set of researchers have
argued that the literature is full of Type I errors (Benjamin et al.,
2018), and thus, the \(\alpha\) value should be shifted lower (i.e.,
\emph{p} \textless{} .005 for statistical significance), an equally
large set of researchers counter this argument as unfounded and weak
(Lakens et al., 2018). We provide multiple sources of evidence
(\emph{p}-values, effect sizes, Bayes Factors, and tests of equivalence)
to determine if differences found are not only statistically
significant, but also practically significant. In our study, we expand
to item randomization for online based surveys, examining the impact on
factor loadings, correlation structure, item means, and total scores
again providing evidence of difference/non-difference from multiple
statistical sources. Finally, we examine these scenarios with a unique
set of scales that have both dichotomous True/False and traditional 1-7
formats to explore how the answer response options might impact any
differences found between randomized and nonrandomized methodologies.

\section{Method}\label{method}

\subsection{Participants}\label{participants}

The sample population consisted of undergraduate students at a large
Midwestern University, placing the approximate age of participants at
around 18-22. Table \ref{tab:demo-table} includes the demographic
information about all datasets. Only two scales were used from each
dataset, as described below. Participants were generally enrolled in an
introductory psychology course that served as a general education
requirement for the university. As part of the curriculum, the students
were encouraged to participate in psychology research programs,
resulting in their involvement in this study. These participants were
given course credit for their participation.

\subsection{Materials}\label{materials}

Of the surveys included within each larger study, two questionnaires
were utilized: the Purpose in Life Questionnaire (PIL; Crumbaugh \&
Maholick, 1964) and the Life Purpose Questionnaire (LPQ; Hutzell, 1988).

\subsubsection{The Purpose in Life
Questionnaire}\label{the-purpose-in-life-questionnaire}

The PIL is a 20-item questionnaire that assesses perceived meaning and
life purpose. Items are structured in a 7-point type response format;
however, each item has different anchoring points that focus on item
content. No items are reverse scored, although, items are presented such
that the 7 point end would be equally presented on the left and right
when answering. Therefore, these items would need to be reverse coded if
computer software automatically codes each item from 1 to 7 in a left to
right format. Total scores are created by summing the items, resulting
in a range of 20 to 140 for the overall score. The reliability reported
for the scale has previously ranged from .70 to .90 (Schulenberg, 2004;
Schulenberg \& Melton, 2010). Previous work on validity for the PIL
showed viable one- and two-factor models, albeit factor loadings varied
across publications (see Schulenberg \& Melton, 2010 for a summary), and
these fluctuating results lead to the development of a 4-item PIL short
form (Schulenberg, Schnetzer, \& Buchanan, 2011).

\subsubsection{Life Purpose
Questionnaire}\label{life-purpose-questionnaire}

The LPQ was modeled after the full 20-item PIL questionnaire, also
measuring perceived meaning and purpose in life. The items are
structured in a true/false response format, in contrast to the 1-7
response format found on the PIL. Each question is matched to the PIL
with the same item content, altering the question to create binary
answer format. After reverse coding, scoring a zero on an item would
indicate low meaning, while scoring a one on an item would indicate high
meaning. A total score is created by summing item scores, resulting in a
range from 0 to 20. In both scales, higher scores indicated greater
perceived meaning in life. Reliability reported for this scale is
usually in the .80 range (Melton \& Schulenberg, 2008; Schulenberg,
2004).

These two scales were selected because they contained the same item
content with differing response formats, which would allow for cross
comparisons between results for each scale.

\subsection{Procedure}\label{procedure}

The form of administration was of interest to this study, and therefore,
two formats were included: computerized administration in nonrandom
order and computerized administration with a randomized question order.
Computerized questionnaires were available for participants to access
electronically, and they were allowed to complete the experiment from
anywhere with the Internet through Qualtrics. To ensure participants
were properly informed, both an introduction and a debriefing were
included within the online form. Participants were randomly assigned to
complete a nonrandomized or randomized version of the survey.
Nonrandomized questionnaires followed the original scale question order,
consistent with paper delivery format. A different group of participants
were given each question in a randomized order within each scale (i.e.,
all PIL and LPQ questions will still grouped together on one page). The
order of administration of the two scales was randomized across
participants for both groups. Once collected, the results were then
amalgamated into a database for statistical analysis.

\section{Results}\label{results}

\subsection{Hypotheses and Data-Analytic
Plan}\label{hypotheses-and-data-analytic-plan}

Computer forms were analyzed by randomized and nonrandomized groups to
examine the impact of randomization on equivalence through correlation
matrices, factor loadings, item means, and total scores. We expected to
find that these forms may potentially vary across correlation structure
and item means, which would indicate differences in reactivity and item
context to questions (i.e., item four always has item three as a
precursor on a nonrandom form, while item four may have a different set
of answers when prefaced with other questions; Knowles et al., 1992).
Factor loadings were assessed to determine if differences in
randomization caused a change in loadings (Buchanan et al., 2005).
However, we did not predict if these values would be different, as
previous research indicates that participants may have a change in
context with a different item order, but this change may not impact the
items relationship with the factor. Last, we examined total scores;
however, it was unclear if these values would change. A difference in
item means may result in changes in total scores, but may also result in
no change if some item means decrease, while others increase.

Each hypothesis was therefore tested using four dependent measures.
First, we examined the correlation matrix for each type of delivery and
compared the matrices to each other by using the \emph{cortest.mat}
function in the \emph{psych} package (Revelle, 2017). This test provides
a \(\chi^2\) value that represents the difference between a pair of
correlation matrices. If this value was significant, we followed up by
exploring the differences between correlations individually using
Fisher's r to z transformation. Each pair of correlations (i.e., random
\(r_{12}\) versus nonrandom \(r_{12}\)) was treated as an independent
correlation and the difference between them was calculated by:

\[
Z_{difference} = \frac{(Z_{1} - Z_{2})} { \sqrt{ \frac{1} {N_{1} - 3} + \frac{1} {N_{2} - 3}} }
\] Critical \(Z_{difference}\) was considered +/- 1.96 for this
analysis, and all values are provided online on at
\url{https://osf.io/gvx7s/}. This manuscript was written in \emph{R}
markdown with the \emph{papaja} package (Aust \& Barth, 2017), and this
document, the data, and all scripts used to calculate our statistics are
available on the OSF page.

We then conducted an exploratory factor analysis on both scales using
one-factor models to examine the loading of each item on its latent
trait. The PIL factor structure is contested (Strack \& Schulenberg,
2009) with many suggestions as to latent structure for one- and
two-factor models. The LPQ has seen less research on factor structure
(Schulenberg, 2004). This paper focused on loadings on one global latent
trait to determine if the manipulation of delivery impacted factor
loadings. We used a one-factor model and included all questions to focus
on the loadings, rather than the factor structure. The analysis was
performed using the \emph{psych} package in \emph{R} with maximum
likelihood estimation. The LPQ factor analysis used tetrachoric
correlation structure to control for the dichotomous format of the
scale, rather than traditional Pearson correlation structure. The
loadings were then compared using a matched dependent \emph{t}-test
(i.e., item one to item one, item two to item two) to examine
differences between nonrandomized and randomized computer samples.

Next, item averages were calculated across all participants for each
item. These 20 items were then compared in a matched dependent
\emph{t}-test to determine if delivery changed the mean of the item on
the PIL or LPQ. While correlation structure elucidates the varying
relations between items, we may still find that item averages are pushed
one direction or another by a change in delivery and still maintain the
same correlation between items. If this test was significant, we
examined the individual items across participants for large effect
sizes, as the large sample sizes in this study would create significant
\emph{t}-test follow ups.

Last, the total scores for each participant were compared across
delivery type using an independent \emph{t}-test. Item analyses allow a
focus on specific items that may show changes, while total scores allow
us to investigate if changes in delivery alter the overall score that is
used in other analyses or possible clinical implications. For analyses
involving \emph{t}-tests, we provide multiple measures of evidentiary
value so that researchers can weigh the effects of randomization on
their own criterion. Recent research on \(\alpha\) criteria has shown
wide disagreement on the usefulness of \emph{p}-values and set cut-off
scores (Benjamin et al., 2018; Lakens et al., 2018). Therefore, we
sought to provide traditional null hypothesis testing results
(\emph{t}-tests, \emph{p}-values) and supplement these values with
effect sizes (\emph{d} and non-central confidence intervals, Buchanan,
Valentine, \& Scofield, 2017; Cumming, 2014; Smithson, 2001), Bayes
Factors (Kass \& Raftery, 1995; Morey \& Rouder, 2015), and one-sided
tests of equivalence (TOST, Cribbie, Gruman, \& Arpin-Cribbie, 2004;
Lakens, 2017; Rogers, Howard, \& Vessey, 1993; Schuirmann, 1987).

For dependent \emph{t}-tests, we used the average standard deviation of
each group as the denominator for \emph{d} calculation as follows
(Cumming, 2012):

\[
d_{av} = \frac {(M_1 -  M_2) } { \frac{SD_1 + SD_2 } {2} }
\] This effect size for repeated measures was used instead of the
traditional \(d_z\) formula, wherein mean differences are divided by the
standard deviation of the difference scores (Lakens, 2013). The
difference scores standard deviation is often much smaller than the
average of the standard deviations of each level, which can create an
upwardly biased effect size (Cumming, 2014). This bias can lead
researchers to interpret larger effects for a psychological phenomenon
than actually exist. Lakens (2013) recommends using \(d_{av}\) over
\(d_z\) because \(d_z\) can overestimate the effect size (see also,
Dunlap, Cortina, Vaslow, \& Burke, 1996) and \(d_{av}\) can be more
comparable to between subjects designs \emph{d} values. For independent
\emph{t}-tests, we used the \(d_s\) formula (Cohen, 1988):

\[
d_s = \frac {(M_1 -  M_2) } { \sqrt{\frac{(N_1-1)SD_1 + (N_2-1)SD_2 } {N_1+N_2-2}} }
\]

The normal frequentist approach (NHST) focuses largely on significance
derived from \emph{p}-values while Bayesian approaches allow for the
calculation of Bayes Factors that provide estimates of the support for
one model as compared to another (Dienes, 2014; Wagenmakers, 2007). NHST
methods traditionally involve two competing hypotheses: a null or nil
hypothesis of no change between groups (Cohen, 1994) and an alternative
or research hypothesis of change between groups, as a mish-mash of
Fisherian and Neyman-Pearson methods. However, one limitation to this
approach is the inability to support the null hypothesis (Gallistel,
2009). Within a Bayesian framework, one focuses on the uncertainty or
probability of phenomena, including the likelihood of no differences
between groups (Lee \& Wagenmakers, 2014). Again, we can create two
models: one of the null where both groups arise from the distribution
with given parameters and one of the alternative where each group arises
from different distributions with their own unique parameters. For both
these models, before seeing the data, the researcher decides what they
believe the distributions of these parameters look like before creating
prior distributions. When data is collected, it is used to inform and
update these prior distributions creating posterior distributions.
Because the Bayesian framework focuses on updating previous beliefs with
the data collected to form new beliefs, any number of hypotheses may be
tested (for a humorous example, see Wagenmakers, Morey, \& Lee, 2016). A
Bayesian version of significance testing may be calculated by using
model comparison through Bayes Factors (Etz \& Wagenmakers, 2017; Kass
\& Raftery, 1995; Rouder, Speckman, Sun, Morey, \& Iverson, 2009). Bayes
Factors are calculated as a ratio of the marginal likelihood of the two
models. Bayes Factors provide a numeric value for how likely one model
is over another model, much like likelihood or odds ratios.

Here, Bayes Factors (BF) are calculated as the marginal likelihood of
the observed data under the alternative hypothesis divided by the
marginal likelihood of the data with the null hypothesis. The resulting
ratio can therefore give evidence to the support of one model as
compared to another, where BF values less than one indicate support for
the null model, values near one indicate both models are equally
supported, and values larger than one indicate support for the
alternative model. While some researchers have proposed conventions for
BF values to discuss the strength of the evidence (Kass \& Raftery,
1995), we will present these values as a continuum to allow researchers
to make their own decisions (Morey, 2015; Morey \& Rouder, 2015). Using
this Bayesian approach, we are then able to show support for or against
the null model, in contrast to NHST where we can only show support
against the null (Gallistel, 2009).

Specifically, we used the \emph{BayesFactor} package (Morey \& Rouder,
2015) with the recommended default priors that cover a wide range of
data (Ly \& Verhagen, 2016; Morey \& Rouder, 2015; Rouder et al., 2009)
of a Jeffreys prior with a fixed rscale (0.5) and random rscale (1.0).
The choice of prior distribution can heavily influence the posterior
belief, in that uninformative priors allow the data to comprise the
posterior distribution. However, most researchers have a background
understanding of their field, thus, making completely uninformative
priors a tenuous assumption. Because of the dearth of literature in this
field, there is not enough previous information to create a strong prior
distribution, which would suppress the effect of the data on posterior
belief. Therefore, we used the default options in \emph{BayesFactor} to
model this belief.

Using Bayes Factors, we may be able to show evidence of the absence of
an effect. Often, non-significant \emph{p}-values from a NHST analysis
are misinterpreted as evidence for the null hypothesis (Lakens, 2017).
However, we can use the traditional frequentist approach to determine if
an effect is within a set of equivalence bounds. We used the two
one-sided tests approach to specify a range of raw-score equivalence
that would be considered supportive of the null hypothesis (i.e., no
worthwhile effects or differences). TOST are then used to determine if
the values found are outside of the equivalence range. Significant TOST
values indicate that the effects are \emph{within} the range of
equivalence. We used the \emph{TOSTER} package (Lakens, 2017) to
calculate these values, and graphics created from this package can be
found online on our OSF page.

The equivalence ranges are often tested by computing an expected effect
size of negligible range; however, the TOST for dependent \emph{t} uses
\(d_z\), which can overestimate the effect size of a phenomena (Cumming,
2014; Lakens, 2013). Therefore, we calculated TOST tests on raw score
differences to alleviate the overestimation issues. For EFA, we used a
change score of .10 in the loadings, as Comrey and Lee (1992) suggested
loading estimation ranges, such as .32 (poor) to .45 (fair) to .55
(good), and the differences in these ranges are approximately .10 (as
cited in Tabachnick \& Fidell, 2012, p. 654). Additionally, this score
would amount to a small correlation change using traditional guidelines
for interpretation of \emph{r} (Cohen, 1992). For item and total score
differences, we chose a 5\% change in magnitude as the raw score cut off
as a modest raw score change. To calculate that change for total scores,
we used the following formula:

\[
(Max*N_{Questions} - Min*N_{Questions}) * Change
\] Minimum and maximum values indicate the lower and upper end of the
answer choices (i.e., 1 and 7), and change represented the proportion
magnitude change expected. Therefore, for total PIL scores, we proposed
a change in 6 points to be significant, while LPQ scores would change 1
point to be a significant change. For item analyses, we divided the
total score change by the number of items to determine how much each
item should change to impact the total score a significant amount (PIL =
0.30, LPQ = .05).

As discussed in the introduction, another approach to measuring
equivalence would be through a MGCFA framework, analyzing measurement
invariance. Those analyses were calculated as a supplement to the
analyses described above and a summary is provided online. The original
goal of this project was to calculate potential reactivity to item order
through analyses that would be accessible to most researchers using
questionnaires in their research. MGCFA requires not only specialized
knowledge, but also specific software and the associated learning curve.
We used \emph{R} in our analyses, however, all analyses presented can be
recreated with free software. The writers of \emph{BayesFactor} have
published online calculators for their work at
\url{http://pcl.missouri.edu/bayesfactor}, and BF values are also
available in \emph{JASP} (JASP Team, 2018). The TOST analyses may be
calculated using an Excel spreadsheet available from the author at
\url{https://osf.io/qzjaj/} or as an add-in module in the program
\emph{jamovi} (project, 2018). Both \emph{JASP} and \emph{jamovi} are
user friendly programs that researchers familiar with point and click
software like Excel or SPSS will be able to use with ease.

\subsection{Data Screening}\label{data-screening}

Each dataset was analyzed separately by splitting on scale and
randomization, and first, all data were screened for accuracy and
missing data. Participants with more than 5\% missing data (i.e., 2 or
more items) were excluded, as Tabachnick and Fidell (2012) have
suggested that 5\% or less of missing data may be safely filled in with
minimal effects on hypothesis testing. Table \ref{tab:demo-table}
indicates the number of participants who were excluded for each set as a
function of: 1) missing more than 5\% of their data, 2) were missing
data due to experimenter error (i.e., some versions of the PIL did not
have one item, and these were excluded), or 3) missing values for the
LPQ include participants who did not see this scale in some original
rounds of the survey. Because we were examining context item-order
effects, it did not seem prudent to include participants who were
missing larger portions of their data, as it would be unclear if their
context was the same as participants who did complete the entire survey.
Our final sample sizes, as shown in Table \ref{tab:demo-table} remained
sufficiently large for analyses described below.

For participants with less than 5\% missing data, we used the
\emph{mice} package in \emph{R} to impute multiple datasets with those
points filled in (Van Buuren \& Groothuis-Oudshoorn, 2011). For the PIL
randomized, \emph{n} = 43 data points were imputed, \emph{n} = 60 for
the nonrandomized PIL, \emph{n} = 15 for the randomized LPQ, and
\emph{n} = 33 for the nonrandomized LPQ. The advantage to using the
\emph{mice} package is the automatic estimation of missing data points
based on the data type (i.e., 1-7 versus binary), rather than simple
mean estimation. The default number of imputations is five, and one was
selected to combine with the original dataset for analyses described
below.

Next, each dataset was examined for multivariate outliers using
Mahalanobis distance. As described in Tabachnick and Fidell (2012),
Mahalanobis values were calculated for each participant based on their
answer choice patterns for each of the twenty questions. These \emph{D}
values are compared to a \(\chi^2\)(20)\(_{p<.001}\) = 45.31, and
observations with \emph{D} values greater than this score were counted
as outliers. This analysis is similar to using a \emph{z}-score
criterion of three standard deviations away from the mean. Each dataset
was then screened for multivariate assumptions of additivity, linearity,
normality, homogeneity, and homoscedasticity. While some data skew was
present, large sample sizes allowed for the assumption of normality of
the sampling distribution. Information about the number of excluded data
points and final sample size in each step is presented in Table
\ref{tab:demo-table}.

\begin{table}[tbp]
\begin{center}
\begin{threeparttable}
\caption{\label{tab:demo-table}Demographic and Data Screening Information}
\small{
\begin{tabular}{lccccccc}
\toprule
Group & \multicolumn{1}{c}{Female} & \multicolumn{1}{c}{White} & \multicolumn{1}{c}{Age (SD)} & \multicolumn{1}{c}{Original N} & \multicolumn{1}{c}{Missing N} & \multicolumn{1}{c}{Outlier N} & \multicolumn{1}{c}{Final N}\\
\midrule
PIL Random & 61.6 & 81.1 & 19.50 (2.93) & 1462 & 333 & 59 & 1070\\
PIL Not Random & 54.1 & 78.6 & 19.68 (3.58) & 915 & 51 & 36 & 828\\
LPQ Random & - & - & - & 1462 & 555 & 24 & 883\\
LPQ Not Random & - & - & - & 915 & 150 & 16 & 749\\
\bottomrule
\addlinespace
\end{tabular}
}
\begin{tablenotes}[para]
\textit{Note.} Participants took both the PIL and LPQ scale, therefore, random and not random demographics are the same. Not every participant was given the LPQ, resulting in missing data for those subjects. Several PIL participants were removed because they were missing an item on their scale.
\end{tablenotes}
\end{threeparttable}
\end{center}
\end{table}

\subsection{PIL Analyses}\label{pil-analyses}

\subsubsection{Correlation Matrices}\label{correlation-matrices}

The correlation matrices for the randomized and nonrandomized versions
of the PIL were found to be significantly different, \(\chi^2\)(380) =
784.84, \emph{p} \textless{} .001. The \emph{Z} score differences were
examined, and 32 correlations were different across the possible 190
tests. A summary of differences can be found in Table
\ref{tab:cor-table-pil}. For each item, the total number of differences
was calculated, as shown in column two, and those specific items are
listed in column three. The last two columns summarize the directions of
these effects. Positive \emph{Z}-scores indicated stronger correlations
between nonrandomized items, while negative \emph{Z}-scores indicated
stronger correlations for randomized items (summarized in the last
column). Two items had strong context effects (i.e., impacted many
items), item 2 \emph{exciting life} and item 15 \emph{prepared for
death}. Interestingly, the impact is the reverse for these two items, as
item 2 showed stronger relationships to items when randomized, while
item 15 showed stronger relationships to items when nonrandomized.

\begin{table}[tbp]
\begin{center}
\begin{threeparttable}
\caption{\label{tab:cor-table-pil}Correlation Matrices Results by Item for the PIL}
\small{
\begin{tabular}{lcccc}
\toprule
Item & Differences & Items Changed & Direction of Change & Stronger Randomized\\
\midrule
1 & 3 & 2, 12, 15 & 2 Negative; 1 Positive & 2 \& 12\\
2 & 9 & 1, 3, 4, 8, 9, 
15, 18, 19, 20 & 8 Negative; 1 Positive & 1, 3, 4, 8, 9, 
18, 19, 20\\
3 & 1 & 2 & 1 Negative & 2\\
4 & 2 & 2, 15 & 1 Negative; 1 Positive & 2\\
5 & 2 & 9, 15 & 1 Negative; 1 Positive & 9\\
6 & 2 & 12, 15 & 2 Positive & N/A\\
7 & 2 & 17, 19 & 2 Positive & N/A\\
8 & 1 & 2 & 1 Negative & 2\\
9 & 3 & 2, 5, 15 & 2 Negative; 1 Positive & 2 \& 5\\
10 & 2 & 12, 15 & 2 Positive & N/A\\
11 & 3 & 12, 15, 20 & 3 Positive & N/A\\
12 & 6 & 1, 6, 10, 11, 14, 20 & 2 Negative; 4 Positive & 1 \& 14\\
13 & 0 & N/A & N/A & N/A\\
14 & 2 & 12, 18 & 2 Negative & 12 \& 18\\
15 & 10 & 1, 2, 4, 5, 6, 9, 
10, 11, 17, 19 & 10 Positive & N/A\\
16 & 0 & N/A & N/A & N/A\\
17 & 4 & 7, 15, 18, 19 & 4 Positive & N/A\\
18 & 3 & 2, 14, 17 & 2 Negative; 1 Positive & 2 \& 14\\
19 & 5 & 2, 7, 15, 17, 20 & 1 Negative; 4 Positive & 2\\
20 & 4 & 2, 11, 12, 19 & 1 Negative; 3 Positive & 2\\
\bottomrule
\end{tabular}
}
\end{threeparttable}
\end{center}
\end{table}

\subsubsection{Factor Loadings}\label{factor-loadings}

Table \ref{tab:Ptable} includes the factor loadings from the one-factor
EFA analysis. These loadings were compared using a dependent
\emph{t}-test matched on item, and they were not significantly
different, \(M_d = 0.00\), 95\% CI \([-0.02\), \(0.03]\),
\(t(19) = 0.25\), \(p = .802\). The effect size for this test was
correspondingly negligible, \(d_{av}\) = -0.02 95\% CI {[}-0.45,
0.42{]}. The TOST test was significant for both the lower, \emph{t}(19)
= 0.19, \emph{p} \textless{} .001 and the upper bound, \emph{t}(19) =
-0.70, \emph{p} \textless{} .001. This result indicated that the change
score was within the confidence band of expected negligible changes.
Lastly, the BF for this test was 0.24, which indicated support for the
null model.

\subsubsection{Item Means}\label{item-means}

Table \ref{tab:Ptable} includes the means and standard deviation of each
item from the PIL scale. The item means were compared using a dependent
\emph{t}-test matched on item. Item means were significantly different
\(M_d = -0.07\), 95\% CI \([-0.13\), \(-0.02]\), \(t(19) = -2.91\),
\(p = .009\). The effect size for this difference was small, \(d_{av}\)
= -0.16 95\% CI {[}-0.60, 0.29{]}. Even though the \emph{t}-test was
significant, the TOST test indicated that the difference was within the
range of a 5\% percent change in item means (0.30). The TOST test for
lower bound, \emph{t}(19) = -1.57, \emph{p} \textless{} .001 and the
upper bound, \emph{t}(19) = -4.26, \emph{p} \textless{} .001, suggested
that the significant \emph{t}-test may be not be interpreted as a
meaningful change on the item means. The BF value for this test
indicated 6.86, which is often considered weak evidence for the
alternative model. Here, we find mixed results, indicating that
randomization may change item means for the PIL.

\subsubsection{Total Scores}\label{total-scores}

\begin{table}[tbp]
\begin{center}
\begin{threeparttable}
\caption{\label{tab:Ptable}Item Statistics for the PIL Scale}
\begin{tabular}{lcccccc}
\toprule
Item & \multicolumn{1}{c}{FL-R} & \multicolumn{1}{c}{FL-NR} & \multicolumn{1}{c}{M-R} & \multicolumn{1}{c}{M-NR} & \multicolumn{1}{c}{SD-R} & \multicolumn{1}{c}{SD-NR}\\
\midrule
1 & .667 & .638 & 4.829 & 4.806 & 1.279 & 1.278\\
2 & .679 & .572 & 4.929 & 4.600 & 1.437 & 1.452\\
3 & .685 & .671 & 5.815 & 5.732 & 1.124 & 1.101\\
4 & .839 & .847 & 5.673 & 5.655 & 1.300 & 1.285\\
5 & .639 & .574 & 4.666 & 4.407 & 1.496 & 1.497\\
6 & .674 & .685 & 5.425 & 5.338 & 1.308 & 1.400\\
7 & .424 & .439 & 6.172 & 6.081 & 1.207 & 1.373\\
8 & .626 & .596 & 5.014 & 5.011 & 1.092 & 1.139\\
9 & .823 & .796 & 5.355 & 5.327 & 1.176 & 1.198\\
10 & .723 & .764 & 5.202 & 5.156 & 1.502 & 1.543\\
11 & .775 & .796 & 5.222 & 5.165 & 1.629 & 1.621\\
12 & .604 & .649 & 4.496 & 4.527 & 1.570 & 1.600\\
13 & .429 & .403 & 5.745 & 5.738 & 1.244 & 1.216\\
14 & .449 & .421 & 5.431 & 5.239 & 1.377 & 1.547\\
15 & .081 & .211 & 4.376 & 4.149 & 1.941 & 1.884\\
16 & .547 & .554 & 5.099 & 5.266 & 1.983 & 1.861\\
17 & .720 & .735 & 5.422 & 5.399 & 1.393 & 1.404\\
18 & .483 & .501 & 5.387 & 5.302 & 1.474 & 1.593\\
19 & .678 & .721 & 4.879 & 4.907 & 1.412 & 1.455\\
20 & .782 & .810 & 5.343 & 5.210 & 1.314 & 1.289\\
\bottomrule
\addlinespace
\end{tabular}
\begin{tablenotes}[para]
\textit{Note.} FL = Factor Loadings, M = Mean, SD = Standard Deviation, R = Random, NR = Not Random
\end{tablenotes}
\end{threeparttable}
\end{center}
\end{table}

Total scores were created by summing the items for each participant
across all twenty PIL questions. The mean total score for nonrandomized
testing was \emph{M} = 103.01 (\emph{SD} = 18.29) with excellent
reliability (\(\alpha\) = .93), while the mean for randomizing testing
was \emph{M} = 104.48 (\emph{SD} = 17.81) with excellent reliability
(\(\alpha\) = .92). The total score difference was examined with an
independent \emph{t}-test and was not significant, \(t(1,896) = -1.76\),
\(p = .079\). The effect size for this difference was negligible,
\(d_{av}\) = -0.08 95\% CI {[}-0.17, 0.29{]}. We tested if scores were
changed by 5\% (6.00 points), and the TOST test indicated that the
lower, \emph{t}(1897) = 5.43, \emph{p} \textless{} .001 and the upper
bound, \emph{t}(1897) = -8.95, \emph{p} \textless{} .001 were within
this area of null change. The BF results also supported the null model,
0.25.

\subsection{LPQ Analyses}\label{lpq-analyses}

\subsubsection{Correlation Matrices}\label{correlation-matrices-1}

Mirroring the results for the PIL, the correlation matrices for the
randomized and nonrandomized versions of the LPQ were significantly
different, \(\chi^2\)(380) = 681.72, \emph{p} \textless{} .001. Less
differences in correlation were found as compared to the PIL, only 19
out of the possible 190 combinations. The differences are summarized in
Table \ref{tab:cor-table-lpq}. Most of the items affected one to four
other items with item 13 \emph{reliable person} showing the largest
number of differences in correlation. All these changes were positive,
meaning the correlations were larger for nonrandomized versions.

\begin{table}[tbp]
\begin{center}
\begin{threeparttable}
\caption{\label{tab:cor-table-lpq}Correlation Matrices Results by Item for the LPQ}
\small{
\begin{tabular}{lcccc}
\toprule
Item & Differences & Items Changed & Direction of Change & Stronger Randomized\\
\midrule
1 & 3 & 11, 13, 18 & 1 Negative; 2 Positive & 18\\
2 & 1 & 6 & 1 Positive & N/A\\
3 & 1 & 8 & 1 Negative & 8\\
4 & 0 & N/A & N/A & N/A\\
5 & 2 & 6, 11 & 2 Positive & N/A\\
6 & 2 & 2, 5 & 2 Positive & N/A\\
7 & 3 & 13, 18, 20 & 3 Positive & N/A\\
8 & 2 & 3, 20 & 2 Negative & 3 \& 20\\
9 & 2 & 11, 13 & 2 Positive & N/A\\
10 & 1 & 20 & 1 Positive & N/A\\
11 & 4 & 1, 5, 9, 12 & 4 Positive & N/A\\
12 & 2 & 11, 13 & 2 Positive & N/A\\
13 & 6 & 1, 7, 12, 15, 16 & 6 Positive & N/A\\
14 & 0 & N/A & N/A & N/A\\
15 & 0 & N/A & N/A & N/A\\
16 & 1 & 13 & 1 Positive & N/A\\
17 & 1 & 13 & 1 Positive & N/A\\
18 & 3 & 1, 7, 20 & 1 Negative; 2 Positive & 1\\
19 & 0 & N/A & N/A & N/A\\
20 & 4 & 7, 8, 10, 18 & 1 Negative; 3 Positive & 8\\
\bottomrule
\end{tabular}
}
\end{threeparttable}
\end{center}
\end{table}

\subsubsection{Factor Loadings}\label{factor-loadings-1}

Table \ref{tab:Ltable} includes the factor loadings from the one-factor
EFA analysis using tetrachoric correlations. The loadings from
randomized and nonrandomized versions were compared using a dependent
\emph{t}-test matched on item, which indicated they were not
significantly different, \(M_d = 0.01\), 95\% CI \([-0.02\), \(0.04]\),
\(t(19) = 0.97\), \(p = .344\). The difference found for this test was
negligible, \(d_{av}\) = -0.07 95\% CI {[}-0.50, 0.37{]}. The TOST test
examined if any change was within .10 change, as described earlier. The
lower, \emph{t}(19) = -0.52, \emph{p} \textless{} .001 and the upper
bound, \emph{t}(19) = -1.42, \emph{p} \textless{} .001 were both
significant, indicating that the change was within the expected change.
Further, in support of the null model, the BF was 0.34.

\subsubsection{Item Means}\label{item-means-1}

Means and standard deviations of each item are presented in Table
\ref{tab:Ltable}. We again matched items and tested if there was a
significant change using a dependent \emph{t}-test. The test was not
significant, \(M_d = 0.00\), 95\% CI \([-0.02\), \(0.02]\),
\(t(19) = 0.26\), \(p = .797\), and the corresponding effect size
reflects how little these means changed, \(d_{av}\) = 0.01 95\% CI
{[}-0.42, 0.45{]}. Using a 5\% change criterion, items were tested to
determine if they changed less than (0.05). The TOST test indicated both
lower, \emph{t}(19) = 0.48, \emph{p} \textless{} .001 and the upper
bound, \emph{t}(19) = 0.04, \emph{p} \textless{} .001, were within the
null range. The BF also supported the null model, 0.24.

\subsubsection{Total Scores}\label{total-scores-1}

\begin{table}[tbp]
\begin{center}
\begin{threeparttable}
\caption{\label{tab:Ltable}Item Statistics for the LPQ Scale}
\begin{tabular}{lcccccc}
\toprule
Item & \multicolumn{1}{c}{FL-R} & \multicolumn{1}{c}{FL-NR} & \multicolumn{1}{c}{M-R} & \multicolumn{1}{c}{M-NR} & \multicolumn{1}{c}{SD-R} & \multicolumn{1}{c}{SD-NR}\\
\midrule
1 & .675 & .682 & .567 & .613 & .496 & .487\\
2 & .900 & .870 & .754 & .760 & .431 & .428\\
3 & .503 & .394 & .864 & .844 & .343 & .363\\
4 & .730 & .685 & .908 & .868 & .289 & .339\\
5 & .687 & .682 & .419 & .507 & .494 & .500\\
6 & .502 & .555 & .638 & .582 & .481 & .494\\
7 & .193 & .286 & .775 & .810 & .418 & .392\\
8 & .555 & .471 & .482 & .467 & .500 & .499\\
9 & .856 & .911 & .810 & .781 & .393 & .414\\
10 & .592 & .620 & .635 & .646 & .482 & .478\\
11 & .636 & .760 & .727 & .761 & .446 & .427\\
12 & .687 & .758 & .787 & .752 & .410 & .432\\
13 & .314 & .399 & .965 & .911 & .184 & .286\\
14 & .486 & .486 & .762 & .769 & .426 & .422\\
15 & .046 & .102 & .323 & .395 & .468 & .489\\
16 & .700 & .707 & .863 & .872 & .344 & .335\\
17 & .514 & .502 & .847 & .814 & .360 & .389\\
18 & .558 & .511 & .830 & .828 & .376 & .378\\
19 & .675 & .717 & .463 & .497 & .499 & .500\\
20 & .644 & .618 & .721 & .712 & .449 & .453\\
\bottomrule
\addlinespace
\end{tabular}
\begin{tablenotes}[para]
\textit{Note.} FL = Factor Loadings, M = Mean, SD = Standard Deviation, R = Random, NR = Not Random
\end{tablenotes}
\end{threeparttable}
\end{center}
\end{table}

LPQ total scores were created by summing the items for each participant.
The mean total score for randomized testing was \emph{M} = 14.14
(\emph{SD} = 4.01), with good reliability (\(\alpha\) = .82), and the
mean for nonrandomized testing was \emph{M} = 14.19 (\emph{SD} = 4.22)
and good reliability (\(\alpha\) = .84). An independent \emph{t}-test
indicated that testing did not change the total score,
\(t(1,630) = 0.23\), \(p = .819\). The effect size for this difference
was negligible, \(d_{av}\) = 0.01 95\% CI {[}-0.09, 0.45{]}. The TOST
test indicated that the scores were within a 5\% (1.00 points) change,
lower: \emph{t}(1627) = 5.13, \emph{p} \textless{} .001 and upper:
\emph{t}(1627) = -4.67, \emph{p} \textless{} .001. The BF results were
in support of the null model, 0.06.

\section{Discussion}\label{discussion}

As technology has advanced, initial research questioned the validity of
online assessments versus paper assessments. With further investigation,
several researchers discovered equivalence with regard to computer
surveys compared with paper surveys (Deutskens et al., 2006; Lewis et
al., 2009). However, with the addition of technology, Fang et al.
(2012a) suggested that individuals respond with more extreme scores in
online surveys than in-person surveys due to the social-desirability
effect. Research on equivalence is mixed in results for paper and
computer, and our work is a first-step on examining survey equivalence
on an individual item-level for different forms of computer delivery.
The findings from the current study are similar to those of Knowles et
al. (1992), in that we found differences in correlation matrices when
items were randomized versus nonrandomized. These differences may be
attributed to the context of the items when randomized, as described by
Salancik and Brand (1992) When viewed through a meaning-change (Knowles
et al., 1992) or integration model (Rogers, 1974; Tourangeau \&
Rasinski, 1988), these differences may indicate that the context and
background knowledge are shifting based on the order of the items
presented.

As items showed these order context effects, randomization may present a
way to combat those effects where the context of items is equalized
across participants. However, it is important to show that randomization
does not change the relationship of items with that underlying factor,
rather just the context in which these items are presented. In both the
PIL and LPQ scales, the factor loadings were found to be equivalent with
results supporting the null hypothesis. For the PIL, we did find support
for differences in item means using \emph{p}-value criterion and Bayes
Factor analyses. However, the effect size was small, meaning the
differences were potentially not as meaningful as the \emph{p}-values
and BF analyses posit, in addition to considering the evidentiary values
of the two one-sided tests, which supported the null range of expected
values. Potentially, the small difference in item means was due to
fluctuating context and order effects, with more change possible using a
1 to 7 item answer format (i.e., more possible range of answer change).
The LPQ item means were not found to differ, and the correlational
analysis showed less items changed in contrast to the PIL analysis.
Finally, the total scores showed equivalence between randomization and
nonrandomization which suggested that total scales were not considerably
impacted with or without randomization of items. The match between
results for two types of answer methodologies implied that randomization
can be applied across a variety of scale types with similar effects.

Since the PIL and LPQ analyses predominately illustrated support for
null effects of randomization, item randomization of scales is of
practical use when there are potential concerns about item order and
context effects described by the meaning-change hypotheses. Subject
matter experts are usually involved in the scale development and this
facet of reactivity should be considered in item development and
deployment. Randomization has been largely viewed as virtuous research
practice in terms of sample selection and order of stimuli presentation
for years; now, we must decide if item reactivity earns the same amount
of caution that has been granted to existing research procedures.
Randomization will create a wider range of possible
interpretation-integration context scenarios as participants react and
respond to items. This procedure would even out context effects at the
sample or group level, but individual differences will be present for
each participant.

Since we found equivalence in terms of overall scoring of the PIL and
LPQ, we advise that randomization can be used as a control mechanism, in
addition to the ease of comparison between the scales if one researcher
decided to randomize and one did not. Moreover, these results would
imply that if an individual's total score on the PIL or LPQ is
significantly different on randomized versus nonrandomized
administrations, it is likely due to factors unrelated to delivery.
Future research should investigate if this result is WEIRD (Western,
Educated, Industrialized, Rich, and Democratic), as this study focused
on college-age students in the Midwest (Henrich, Heine, \& Norenzayan,
2010). As Fang et al. (2012b)'s research indicates different effects for
collectivistic cultures, other cultures may show different results based
on randomization. Additionally, one should consider the effects of
potential computer illiteracy on online surveys (Charters, 2004).

A second benefit to using the procedures outlined in this paper to
examine for differences in methodology is the simple implementation of
the analyses. While our analyses were performed in \emph{R}, nearly all
of these analyses can be performed in free point and click software,
such as \emph{jamovi} and \emph{JASP}. Multigroup confirmatory factory
analyses can additionally be used to analyze a very similar set of
questions (Brown, 2006); however, multigroup analyses require a
specialized skill and knowledge set. Bayes Factor and TOST analyses are
included in these free programs and are easy to implement. In this
paper, we have provided examples of how to test the null hypothesis, as
well as ways to include multiple forms of evidentiary value to
critically judge an analysis on facets other than \emph{p}-values
(Valentine et al., 2017).

\newpage

\section{References}\label{references}

\setlength{\parindent}{-0.5in} \setlength{\leftskip}{0.5in}

\hypertarget{refs}{}
\hypertarget{ref-Aust2017}{}
Aust, F., \& Barth, M. (2017). papaja: Create APA manuscripts with R
Markdown. Retrieved from \url{https://github.com/crsh/papaja}

\hypertarget{ref-Bargh1986}{}
Bargh, J. A., \& Pratto, F. (1986). Individual construct accessibility
and perceptual selection. \emph{Journal of Experimental Social
Psychology}, \emph{22}(4), 293--311.
doi:\href{https://doi.org/10.1016/0022-1031(86)90016-8}{10.1016/0022-1031(86)90016-8}

\hypertarget{ref-Benjamin2017}{}
Benjamin, D. J., Berger, J. O., Johannesson, M., Nosek, B. A.,
Wagenmakers, E.-J., Berk, R., \ldots{} Johnson, V. E. (2018). Redefine
statistical significance. \emph{Nature Human Behaviour}, \emph{2}(1),
6--10.
doi:\href{https://doi.org/10.1038/s41562-017-0189-z}{10.1038/s41562-017-0189-z}

\hypertarget{ref-Bethlehem2010}{}
Bethlehem, J. (2010). Selection bias in web surveys. \emph{International
Statistical Review}, \emph{78}(2), 161--188.
doi:\href{https://doi.org/10.1111/j.1751-5823.2010.00112.x}{10.1111/j.1751-5823.2010.00112.x}

\hypertarget{ref-Brown2006}{}
Brown, T. (2006). \emph{Confirmatory factor analysis for applied
research}. New York, NY: The Guilford Press.

\hypertarget{ref-Buchanan2014}{}
Buchanan, E. M., Valentine, K. D., \& Schulenberg, S. E. (2014).
Exploratory and confirmatory factor analysis: Developing the Purpose in
Life Test--Short Form. In P. Bindle (Ed.), \emph{SAGE research methods
cases}. London, United Kingdom: SAGE Publications, Ltd.
doi:\href{https://doi.org/10.4135/978144627305013517794}{10.4135/978144627305013517794}

\hypertarget{ref-Buchanan2017}{}
Buchanan, E. M., Valentine, K. D., \& Scofield, J. E. (2017). MOTE.
Retrieved from \url{https://github.com/doomlab/MOTE}

\hypertarget{ref-Buchanan2005}{}
Buchanan, T., Ali, T., Heffernan, T., Ling, J., Parrott, A., Rodgers,
J., \& Scholey, A. (2005). Nonequivalence of on-line and
paper-and-pencil psychological tests: The case of the prospective memory
questionnaire. \emph{Behavior Research Methods}, \emph{37}(1), 148--154.
doi:\href{https://doi.org/10.3758/BF03206409}{10.3758/BF03206409}

\hypertarget{ref-Buhrmester2011}{}
Buhrmester, M., Kwang, T., \& Gosling, S. D. (2011). Amazon's Mechanical
Turk: A new source of inexpensive, yet high-quality, data?
\emph{Perspectives on Psychological Science}, \emph{6}(1), 3--5.
doi:\href{https://doi.org/10.1177/1745691610393980}{10.1177/1745691610393980}

\hypertarget{ref-Cantrell2007}{}
Cantrell, M. A., \& Lupinacci, P. (2007). Methodological issues in
online data collection. \emph{Journal of Advanced Nursing},
\emph{60}(5), 544--549.
doi:\href{https://doi.org/10.1111/j.1365-2648.2007.04448.x}{10.1111/j.1365-2648.2007.04448.x}

\hypertarget{ref-Charters2004}{}
Charters, E. (2004). New perspectives on popular culture, science and
technology: Web browsers and the new illiteracy. \emph{College
Quarterly}, \emph{7}(1), 1--13.

\hypertarget{ref-Cohen1988}{}
Cohen, J. (1988). \emph{Statistical power analysis for the behavioral
sciences} (2nd ed.). Hillsdale, NJ: Earlbaum.

\hypertarget{ref-Cohen1992a}{}
Cohen, J. (1992). A power primer. \emph{Psychological Bulletin},
\emph{112}(1), 155--159.
doi:\href{https://doi.org/10.1037//0033-2909.112.1.155}{10.1037//0033-2909.112.1.155}

\hypertarget{ref-Cohen1994}{}
Cohen, J. (1994). The earth is round (p \textless{} .05). \emph{American
Psychologist}, \emph{49}(12), 997--1003.
doi:\href{https://doi.org/10.1037/0003-066X.49.12.997}{10.1037/0003-066X.49.12.997}

\hypertarget{ref-Comrey1992}{}
Comrey, A. L., \& Lee, H. B. (1992). \emph{A first course in factor
analysis} (Second.). Psychology Press.

\hypertarget{ref-Cook2000}{}
Cook, C., Heath, F., \& Thompson, R. L. (2000). A meta-analysis of
response rates in Web- or Internet-based surveys. \emph{Educational and
Psychological Measurement}, \emph{60}(6), 821--836.
doi:\href{https://doi.org/10.1177/00131640021970934}{10.1177/00131640021970934}

\hypertarget{ref-Cribbie2004}{}
Cribbie, R. A., Gruman, J. A., \& Arpin-Cribbie, C. A. (2004).
Recommendations for applying tests of equivalence. \emph{Journal of
Clinical Psychology}, \emph{60}(1), 1--10.
doi:\href{https://doi.org/10.1002/jclp.10217}{10.1002/jclp.10217}

\hypertarget{ref-Cronk2002}{}
Cronk, B. C., \& West, J. L. (2002). Personality research on the
Internet: A comparison of Web-based and traditional instruments in
take-home and in-class settings. \emph{Behavior Research Methods,
Instruments, \& Computers}, \emph{34}(2), 177--180.
doi:\href{https://doi.org/10.3758/BF03195440}{10.3758/BF03195440}

\hypertarget{ref-Crumbaugh1964}{}
Crumbaugh, J. C., \& Maholick, L. T. (1964). An experimental study in
existentialism: The psychometric approach to Frankl's concept of
noogenic neurosis. \emph{Journal of Clinical Psychology}, \emph{20}(2),
200--207.
doi:\href{https://doi.org/10.1002/1097-4679(196404)20:2\%3C200::AID-JCLP2270200203\%3E3.0.CO;2-U}{10.1002/1097-4679(196404)20:2\textless{}200::AID-JCLP2270200203\textgreater{}3.0.CO;2-U}

\hypertarget{ref-Cumming2012}{}
Cumming, G. (2012). \emph{Understanding the new statistics: Effect
sizes, confidence intervals, and meta-analysis}. New York, NY:
Routledge.

\hypertarget{ref-Cumming2014}{}
Cumming, G. (2014). The new statistics: Why and how. \emph{Psychological
Science}, \emph{25}(1), 7--29.
doi:\href{https://doi.org/10.1177/0956797613504966}{10.1177/0956797613504966}

\hypertarget{ref-DeLeeuw1988}{}
De Leeuw, E. D., \& Hox, J. J. (1988). The effects of
response-stimulating factors on response rates and data quality in mail
surveys: A test of Dillman's total design method. \emph{Journal of
Official Statistics}, \emph{4}(3), 241--249.

\hypertarget{ref-Deutskens2006}{}
Deutskens, E., Ruyter, K. de, \& Wetzels, M. (2006). An assessment of
equivalence between online and mail surveys in service research.
\emph{Journal of Service Research}, \emph{8}(4), 346--355.
doi:\href{https://doi.org/10.1177/1094670506286323}{10.1177/1094670506286323}

\hypertarget{ref-DeVellis2016a}{}
DeVellis, R. F. (2016). \emph{Scale development: Theory and
applications} (Fourth.). Sage.

\hypertarget{ref-Dienes2014}{}
Dienes, Z. (2014). Using Bayes to get the most out of non-significant
results. \emph{Frontiers in Psychology}, \emph{5}(July), 1--17.
doi:\href{https://doi.org/10.3389/fpsyg.2014.00781}{10.3389/fpsyg.2014.00781}

\hypertarget{ref-Dillman2008}{}
Dillman, D. A., Smyth, J. D., \& Christian, L. M. (2008).
\emph{Internet, mail, and mixed-mode surveys: The tailored design
method} (Third.). Hoboken, NJ: John Wiley \& Sons, Inc.

\hypertarget{ref-Dunlap1996a}{}
Dunlap, W. P., Cortina, J. M., Vaslow, J. B., \& Burke, M. J. (1996).
Meta-analysis of experiments with matched groups or repeated measures
designs. \emph{Psychological Methods}, \emph{1}(2), 170--177.
doi:\href{https://doi.org/10.1037/1082-989X.1.2.170}{10.1037/1082-989X.1.2.170}

\hypertarget{ref-Etz2015}{}
Etz, A., \& Wagenmakers, E.-J. (2017). J. B. S. Haldane's contribution
to the Bayes Factor hypothesis test. \emph{Statistical Science},
\emph{32}(2), 313--329.
doi:\href{https://doi.org/10.1214/16-STS599}{10.1214/16-STS599}

\hypertarget{ref-Fang2012a}{}
Fang, J., Wen, C., \& Pavur, R. (2012a). Participation willingness in
web surveys: Exploring effect of sponsoring corporation's and survey
provider's reputation. \emph{Cyberpsychology, Behavior, and Social
Networking}, \emph{15}(4), 195--199.
doi:\href{https://doi.org/10.1089/cyber.2011.0411}{10.1089/cyber.2011.0411}

\hypertarget{ref-Fang2012}{}
Fang, J., Wen, C., \& Prybutok, V. R. (2012b). An assessment of
equivalence between Internet and paper-based surveys: evidence from
collectivistic cultures. \emph{Quality \& Quantity}, \emph{48}(1),
493--506.
doi:\href{https://doi.org/10.1007/s11135-012-9783-3}{10.1007/s11135-012-9783-3}

\hypertarget{ref-Feldman1988}{}
Feldman, J. M., \& Lynch, J. G. (1988). Self-generated validity and
other effects of measurement on belief, attitude, intention, and
behavior. \emph{Journal of Applied Psychology}, \emph{73}(3), 421--435.
doi:\href{https://doi.org/10.1037//0021-9010.73.3.421}{10.1037//0021-9010.73.3.421}

\hypertarget{ref-Frick2001}{}
Frick, A., Bächtiger, M. T., \& Reips, U.-D. (2001). Financial
incentives, personal information and dropout in online studies. In U.-D.
Reips \& M. Bosnjak (Eds.), \emph{Dimensions of internet science} (pp.
209--219).

\hypertarget{ref-Gallistel2009}{}
Gallistel, C. R. (2009). The importance of proving the null.
\emph{Psychological Review}, \emph{116}(2), 439--53.
doi:\href{https://doi.org/10.1037/a0015251}{10.1037/a0015251}

\hypertarget{ref-Henrich2010}{}
Henrich, J., Heine, S. J., \& Norenzayan, A. (2010). The weirdest people
in the world? \emph{Behavioral and Brain Sciences}, \emph{33}(2-3),
61--83.
doi:\href{https://doi.org/10.1017/S0140525X0999152X}{10.1017/S0140525X0999152X}

\hypertarget{ref-Higgins1983}{}
Higgins, E., \& Lurie, L. (1983). Context, categorization, and recall:
The ``change-of-standard'' effect. \emph{Cognitive Psychology},
\emph{15}(4), 525--547.
doi:\href{https://doi.org/10.1016/0010-0285(83)90018-X}{10.1016/0010-0285(83)90018-X}

\hypertarget{ref-Hox1994}{}
Hox, J. J., \& De Leeuw, E. D. (1994). A comparison of nonresponse in
mail, telephone, and face-to-face surveys. \emph{Quality and Quantity},
\emph{28}(4), 329--344.
doi:\href{https://doi.org/10.1007/BF01097014}{10.1007/BF01097014}

\hypertarget{ref-Hutzell1988}{}
Hutzell, R. (1988). A review of the Purpose in Life Test.
\emph{International Forum for Logotherapy}, \emph{11}(2), 89--101.

\hypertarget{ref-Ilieva2001}{}
Ilieva, J., Baron, S., \& Healy, N. M. (2002). On-line surveys in
international marketing research: Pros and cons. \emph{International
Journal of Market Research}, \emph{44}(3), 361--376.

\hypertarget{ref-JASP2018}{}
JASP Team. (2018). JASP (Version 0.8.6){[}Computer software{]}.
Retrieved from \url{https://jasp-stats.org/}

\hypertarget{ref-Joinson1999}{}
Joinson, A. (1999). Social desirability, anonymity, and Intemet-based
questionnaires. \emph{Behavior Research Methods, Instruments, \&
Computers}, \emph{31}(3), 433--438.
doi:\href{https://doi.org/10.3758/BF03200723}{10.3758/BF03200723}

\hypertarget{ref-Kass1995a}{}
Kass, R. E., \& Raftery, A. E. (1995). Bayes Factors. \emph{Journal of
the American Statistical Association}, \emph{90}(430), 773--795.
doi:\href{https://doi.org/10.2307/2291091}{10.2307/2291091}

\hypertarget{ref-Keppel2004}{}
Keppel, G., \& Wickens, T. (2004). \emph{Design and analysis: A
researcher's handbook} (4th ed.). Upper Saddle River, NJ: Prentice Hall.

\hypertarget{ref-Knowles1992}{}
Knowles, E. S., Coker, M. C., Cook, D. A., Diercks, S. R., Irwin, M. E.,
Lundeen, E. J., \ldots{} Sibicky, M. E. (1992). Order Effects within
personality measures. In N. Schwarz \& S. Sudman (Eds.), \emph{Context
effects in social and psychological research} (pp. 221--236). New York:
Springer-Verlag.

\hypertarget{ref-Lakens2013}{}
Lakens, D. (2013). Calculating and reporting effect sizes to facilitate
cumulative science: A practical primer for t-tests and ANOVAs.
\emph{Frontiers in Psychology}, \emph{4}.
doi:\href{https://doi.org/10.3389/fpsyg.2013.00863}{10.3389/fpsyg.2013.00863}

\hypertarget{ref-Lakens2017a}{}
Lakens, D. (2017). Equivalence tests. \emph{Social Psychological and
Personality Science}, \emph{8}(4), 355--362.
doi:\href{https://doi.org/10.1177/1948550617697177}{10.1177/1948550617697177}

\hypertarget{ref-Lakens2017}{}
Lakens, D., Adolfi, F. G., Albers, C. J., Anvari, F., Apps, M. A. J.,
Argamon, S. E., \ldots{} Zwaan, R. A. (2018). Justify your alpha.
\emph{Nature Human Behaviour}, \emph{2}(3), 168--171.
doi:\href{https://doi.org/10.1038/s41562-018-0311-x}{10.1038/s41562-018-0311-x}

\hypertarget{ref-Lee2012}{}
Lee, M. D., \& Wagenmakers, E.-J. (2014). \emph{Bayesian cognitive
modeling: A practical course} (p. 284). Cambridge University Press.

\hypertarget{ref-Lewis2009}{}
Lewis, I., Watson, B., \& White, K. M. (2009). Internet versus
paper-and-pencil survey methods in psychological experiments:
Equivalence testing of participant responses to health-related messages.
\emph{Australian Journal of Psychology}, \emph{61}(2), 107--116.
doi:\href{https://doi.org/10.1080/00049530802105865}{10.1080/00049530802105865}

\hypertarget{ref-Lord1984}{}
Lord, C. G., Lepper, M. R., \& Preston, E. (1984). Considering the
opposite: A corrective strategy for social judgment. \emph{Journal of
Personality and Social Psychology}, \emph{47}(6), 1231--1243.
doi:\href{https://doi.org/10.1037/0022-3514.47.6.1231}{10.1037/0022-3514.47.6.1231}

\hypertarget{ref-Ly2016}{}
Ly, A., \& Verhagen, J. (2016). Harold Jeffreys's default Bayes factor
hypothesis tests: Explanation, extension, and application in psychology.
\emph{Journal of Mathematical Psychology}, \emph{72}, 19--32.
doi:\href{https://doi.org/10.1016/J.JMP.2015.06.004}{10.1016/J.JMP.2015.06.004}

\hypertarget{ref-MacLeod1992}{}
MacLeod, C., \& Campbell, L. (1992). Memory accessibility and
probability judgments: An experimental evaluation of the availability
heuristic. \emph{Journal of Personality and Social Psychology},
\emph{63}(6), 890--902.
doi:\href{https://doi.org/10.1037//0022-3514.63.6.890}{10.1037//0022-3514.63.6.890}

\hypertarget{ref-Media2016}{}
Media. (2016). \emph{The Total Audience Report: Q1 2016}.

\hypertarget{ref-Melton2008}{}
Melton, A. M. A., \& Schulenberg, S. E. (2008). On the measurement of
meaning: Logotherapy's empirical contributions to humanistic psychology.
\emph{The Humanistic Psychologist}, \emph{36}(1), 31--44.
doi:\href{https://doi.org/10.1080/08873260701828870}{10.1080/08873260701828870}

\hypertarget{ref-Meredith1993}{}
Meredith, W. (1993). Measurement invariance, factor analysis and
factorial invariance. \emph{Psychometrika}, \emph{58}(4), 525--543.
doi:\href{https://doi.org/10.1007/BF02294825}{10.1007/BF02294825}

\hypertarget{ref-Meyerson2003}{}
Meyerson, P., \& Tryon, W. W. (2003). Validating Internet research: A
test of the psychometric equivalence of Internet and in-person samples.
\emph{Behavior Research Methods, Instruments, \& Computers},
\emph{35}(4), 614--620.
doi:\href{https://doi.org/10.3758/BF03195541}{10.3758/BF03195541}

\hypertarget{ref-Morey2015c}{}
Morey, R. D. (2015). On verbal categories for the interpretation of
Bayes factors. Retrieved from
\url{http://bayesfactor.blogspot.com/2015/01/on-verbal-categories-for-interpretation.html}

\hypertarget{ref-Morey2015b}{}
Morey, R. D., \& Rouder, J. N. (2015). BayesFactor: Computation of Bayes
Factors for common designs. Retrieved from
\url{https://cran.r-project.org/package=BayesFactor}

\hypertarget{ref-Musch2000}{}
Musch, J., \& Reips, U.-D. (2000). A brief history of web experimenting.
In M. H. Birnbaum (Ed.), \emph{Psychological experiments on the
internet} (pp. 61--87). Elsevier.
doi:\href{https://doi.org/10.1016/B978-012099980-4/50004-6}{10.1016/B978-012099980-4/50004-6}

\hypertarget{ref-Nosek2002}{}
Nosek, B. A., Banaji, M. R., \& Greenwald, A. G. (2002). E-Research:
Ethics, security, design, and control in psychological research on the
Internet. \emph{Journal of Social Issues}, \emph{58}(1), 161--176.
doi:\href{https://doi.org/10.1111/1540-4560.00254}{10.1111/1540-4560.00254}

\hypertarget{ref-Olson2010}{}
Olson, K. (2010). An examination of questionnaire evaluation by expert
reviewers. \emph{Field Methods}, \emph{22}(4), 295--318.
doi:\href{https://doi.org/10.1177/1525822X10379795}{10.1177/1525822X10379795}

\hypertarget{ref-Panter1992}{}
Panter, A. T., Tanaka, J. S., \& Wellens, T. R. (1992). Psychometrics of
order effects. In N. Schwarz \& S. Sudman (Eds.), \emph{Context efects
in social and psychological research} (pp. 249--264). New York:
Springer-Verlag.

\hypertarget{ref-Petty1986}{}
Petty, R. E., \& Cacioppo, J. T. (1986). \emph{Communication and
persuasion: Central and peripheral routes to attitude change}. New York:
Springer-Verlag.

\hypertarget{ref-Posner1978}{}
Posner, M. I. (1978). \emph{Chronometric explorations of mind}.
Hillsdale, NJ: Erlbaum.

\hypertarget{ref-Preacher2003}{}
Preacher, K. J., \& MacCallum, R. C. (2003). Repairing Tom Swift's
electric factor analysis machine. \emph{Understanding Statistics},
\emph{2}(1), 13--43.
doi:\href{https://doi.org/10.1207/S15328031US0201_02}{10.1207/S15328031US0201\_02}

\hypertarget{ref-jamovi2018}{}
project, J. (2018). jamovi (Version 0.8){[}Computer software{]}.
Retrieved from \url{https://www.jamovi.org}

\hypertarget{ref-Reips2002a}{}
Reips, U.-D. (2002). Standards for Internet-based experimenting.
\emph{Experimental Psychology}, \emph{49}(4), 243--256.
doi:\href{https://doi.org/10.1026//1618-3169.49.4.243}{10.1026//1618-3169.49.4.243}

\hypertarget{ref-Reips2012}{}
Reips, U.-D. (2012). Using the Internet to collect data. In \emph{APA
handbook of research methods in psychology, vol 2: Research designs:
Quantitative, qualitative, neuropsychological, and biological.} (Vol. 2,
pp. 291--310). Washington: American Psychological Association.
doi:\href{https://doi.org/10.1037/13620-017}{10.1037/13620-017}

\hypertarget{ref-Revelle2017}{}
Revelle, W. (2017). \emph{psych: Procedures for Psychological,
Psychometric, and Personality Research}. Evanston, Illinois:
Northwestern University. Retrieved from
\url{https://cran.r-project.org/package=psych}

\hypertarget{ref-Rogers1993}{}
Rogers, J. L., Howard, K. I., \& Vessey, J. T. (1993). Using
significance tests to evaluate equivalence between two experimental
groups. \emph{Psychological Bulletin}, \emph{113}(3), 553--565.
doi:\href{https://doi.org/10.1037/0033-2909.113.3.553}{10.1037/0033-2909.113.3.553}

\hypertarget{ref-Rogers1974}{}
Rogers, T. (1974). An analysis of the stages underlying the process of
responding to personality items. \emph{Acta Psychologica}, \emph{38}(3),
205--213.
doi:\href{https://doi.org/10.1016/0001-6918(74)90034-1}{10.1016/0001-6918(74)90034-1}

\hypertarget{ref-Rouder2009}{}
Rouder, J. N., Speckman, P. L., Sun, D., Morey, R. D., \& Iverson, G.
(2009). Bayesian t tests for accepting and rejecting the null
hypothesis. \emph{Psychonomic Bulletin \& Review}, \emph{16}(2),
225--237.
doi:\href{https://doi.org/10.3758/PBR.16.2.225}{10.3758/PBR.16.2.225}

\hypertarget{ref-Salancik1992}{}
Salancik, G. R., \& Brand, J. F. (1992). Context influences on the
meaning of work. In N. Schwarz \& S. Sudman (Eds.), \emph{Context efects
in social and psychological research} (pp. 237--247). New York:
Springer-Verlag.

\hypertarget{ref-Sanou2017}{}
Sanou, B. (2017, July). ICT Facts and Figures 2017. Retrieved from
\url{http://www.itu.int/en/ITU-D/Statistics/Documents/facts/ICTFactsFigures2017.pdf}

\hypertarget{ref-Schuirmann1987}{}
Schuirmann, D. J. (1987). A comparison of the Two One-Sided Tests
Procedure and the power approach for assessing the equivalence of
average bioavailability. \emph{Journal of Pharmacokinetics and
Biopharmaceutics}, \emph{15}(6), 657--680.
doi:\href{https://doi.org/10.1007/BF01068419}{10.1007/BF01068419}

\hypertarget{ref-Schuldt1994}{}
Schuldt, B. a., \& Totten, J. W. (1994). Electronic mail vs. mail survey
response rates. \emph{Marketing Research}, \emph{6}, 36--39.

\hypertarget{ref-Schulenberg2004}{}
Schulenberg, S. E. (2004). A psychometric investigation of logotherapy
measures and the Outcome Questionnaire (OQ-45.2). \emph{North American
Journal of Psychology}, \emph{6}(3), 477--492.

\hypertarget{ref-Schulenberg2010}{}
Schulenberg, S. E., \& Melton, A. M. A. (2010). A confirmatory
factor-analytic evaluation of the purpose in life test: Preliminary
psychometric support for a replicable two-factor model. \emph{Journal of
Happiness Studies}, \emph{11}(1), 95--111.
doi:\href{https://doi.org/10.1007/s10902-008-9124-3}{10.1007/s10902-008-9124-3}

\hypertarget{ref-Schulenberg1999}{}
Schulenberg, S. E., \& Yutrzenka, B. A. (1999). The equivalence of
computerized and paper-and-pencil psychological instruments:
Implications for measures of negative affect. \emph{Behavior Research
Methods, Instruments, \& Computers}, \emph{31}(2), 315--321.
doi:\href{https://doi.org/10.3758/BF03207726}{10.3758/BF03207726}

\hypertarget{ref-Schulenberg2001}{}
Schulenberg, S. E., \& Yutrzenka, B. A. (2001). Equivalence of
computerized and conventional versions of the Beck Depression
Inventory-II (BDI-II). \emph{Current Psychology}, \emph{20}(3),
216--230.
doi:\href{https://doi.org/10.1007/s12144-001-1008-1}{10.1007/s12144-001-1008-1}

\hypertarget{ref-Schulenberg2011}{}
Schulenberg, S. E., Schnetzer, L. W., \& Buchanan, E. M. (2011). The
Purpose in Life Test-Short Form: Development and psychometric support.
\emph{Journal of Happiness Studies}, \emph{12}(5), 861--876.
doi:\href{https://doi.org/10.1007/s10902-010-9231-9}{10.1007/s10902-010-9231-9}

\hypertarget{ref-Smithson2001}{}
Smithson, M. (2001). Correct confidence intervals for various regression
effect sizes and parameters: The importance of noncentral distributions
in computing intervals. \emph{Educational and Psychological
Measurement}, \emph{61}(4), 605--632.
doi:\href{https://doi.org/10.1177/00131640121971392}{10.1177/00131640121971392}

\hypertarget{ref-Smyth2006}{}
Smyth, J. D. (2006). Comparing check-all and forced-choice question
formats in web surveys. \emph{Public Opinion Quarterly}, \emph{70}(1),
66--77.
doi:\href{https://doi.org/10.1093/poq/nfj007}{10.1093/poq/nfj007}

\hypertarget{ref-Steenkamp1998}{}
Steenkamp, J. E. M., \& Baumgartner, H. (1998). Assessing measurement
invariance in cross‐national consumer research. \emph{Journal of
Consumer Research}, \emph{25}(1), 78--107.
doi:\href{https://doi.org/10.1086/209528}{10.1086/209528}

\hypertarget{ref-Strack1987}{}
Strack, F., \& Martin, L. L. (1987). Thinking, judging, and
communicating: A process account of context effects in attitude surveys.
In \emph{Recent research in psychology} (pp. 123--148). Springer, New
York, NY.
doi:\href{https://doi.org/10.1007/978-1-4612-4798-2_7}{10.1007/978-1-4612-4798-2\_7}

\hypertarget{ref-Strack1985}{}
Strack, F., Schwarz, N., \& Gschneidinger, E. (1985). Happiness and
reminiscing: The role of time perspective, affect, and mode of thinking.
\emph{Journal of Personality and Social Psychology}, \emph{49}(6),
1460--1469.
doi:\href{https://doi.org/10.1037//0022-3514.49.6.1460}{10.1037//0022-3514.49.6.1460}

\hypertarget{ref-Strack2009}{}
Strack, K. M., \& Schulenberg, S. E. (2009). Understanding empowerment,
meaning, and perceived coercion in individuals with serious mental
illness. \emph{Journal of Clinical Psychology}, \emph{65}(10),
1137--1148.
doi:\href{https://doi.org/10.1002/jclp.20607}{10.1002/jclp.20607}

\hypertarget{ref-Tabachnick2012}{}
Tabachnick, B. G., \& Fidell, L. S. (2012). \emph{Using multivariate
statistics} (Sixth.). Boston, MA: Pearson.

\hypertarget{ref-Tesser1978}{}
Tesser, A. (1978). Self-generated attitude change. In \emph{Advances in
experimental social psychology} (Vol. 11, pp. 289--338).
doi:\href{https://doi.org/10.1016/S0065-2601(08)60010-6}{10.1016/S0065-2601(08)60010-6}

\hypertarget{ref-Tourangeau1988}{}
Tourangeau, R., \& Rasinski, K. A. (1988). Cognitive processes
underlying context effects in attitude measurement. \emph{Psychological
Bulletin}, \emph{103}(3), 299--314.
doi:\href{https://doi.org/10.1037//0033-2909.103.3.299}{10.1037//0033-2909.103.3.299}

\hypertarget{ref-Tourangeau1999}{}
Tourangeau, R., Rips, L. J., \& Rasinski, K. (1999). \emph{The
psychology of survey response}. Cambridge, UK: Cambridge University
Press.

\hypertarget{ref-Trent2013}{}
Trent, L. R., Buchanan, E., Ebesutani, C., Ale, C. M., Heiden, L.,
Hight, T. L., \ldots{} Young, J. (2013). A measurement invariance
examination of the Revised Child Anxiety and Depression Scale in a
southern sample. \emph{Assessment}, \emph{20}(2), 175--187.
doi:\href{https://doi.org/10.1177/1073191112450907}{10.1177/1073191112450907}

\hypertarget{ref-Tversky1973}{}
Tversky, A., \& Kahneman, D. (1973). Availability: A heuristic for
judging frequency and probability. \emph{Cognitive Psychology},
\emph{5}(2), 207--232.
doi:\href{https://doi.org/10.1016/0010-0285(73)90033-9}{10.1016/0010-0285(73)90033-9}

\hypertarget{ref-Valentine2017}{}
Valentine, K., Buchanan, E., Scofield, J., \& Beauchamp, M. (2017).
\emph{Beyond p-values: Utilizing multiple estimates to evaluate
evidence}. Open Science Framework.
doi:\href{https://doi.org/10.17605/osf.io/9hp7y}{10.17605/osf.io/9hp7y}

\hypertarget{ref-VanBuuren2011}{}
Van Buuren, S., \& Groothuis-Oudshoorn, K. (2011). mice: Multivariate
imputation by chained equations in R. \emph{Journal of Statistical
Software}, \emph{45}(3), 1--67.
doi:\href{https://doi.org/10.18637/jss.v045.i03}{10.18637/jss.v045.i03}

\hypertarget{ref-Wagenmakers2007}{}
Wagenmakers, E.-J. (2007). A practical solution to the pervasive
problems of p values. \emph{Psychonomic Bulletin \& Review},
\emph{14}(5), 779--804.
doi:\href{https://doi.org/10.3758/BF03194105}{10.3758/BF03194105}

\hypertarget{ref-Wagenmakers2016}{}
Wagenmakers, E.-J., Morey, R. D., \& Lee, M. D. (2016). Bayesian
benefits for the pragmatic researcher. \emph{Current Directions in
Psychological Science}, \emph{25}(3), 169--176.
doi:\href{https://doi.org/10.1177/0963721416643289}{10.1177/0963721416643289}

\hypertarget{ref-Webb1966}{}
Webb, E. S., Campbell, D. T., Schwartz, R. D., \& Sechrest, L. (1966).
\emph{Unobtrusive measures: Nonreactive research in the social
sciences}. Chicago: Rand McNally.

\hypertarget{ref-Weigold2013}{}
Weigold, A., Weigold, I. K., \& Russell, E. J. (2013). Examination of
the equivalence of self-report survey-based paper-and-pencil and
internet data collection methods. \emph{Psychological Methods},
\emph{18}(1), 53--70.
doi:\href{https://doi.org/10.1037/a0031607}{10.1037/a0031607}

\hypertarget{ref-Worthington2006}{}
Worthington, R. L., \& Whittaker, T. a. (2006). Scale development
research: A content analysis and recommendations for best practices.
\emph{The Counseling Psychologist}, \emph{34}(6), 806--838.
doi:\href{https://doi.org/10.1177/0011000006288127}{10.1177/0011000006288127}






\end{document}
