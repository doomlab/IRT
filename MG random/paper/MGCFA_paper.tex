\documentclass[english,man]{apa6}

\usepackage{amssymb,amsmath}
\usepackage{ifxetex,ifluatex}
\usepackage{fixltx2e} % provides \textsubscript
\ifnum 0\ifxetex 1\fi\ifluatex 1\fi=0 % if pdftex
  \usepackage[T1]{fontenc}
  \usepackage[utf8]{inputenc}
\else % if luatex or xelatex
  \ifxetex
    \usepackage{mathspec}
    \usepackage{xltxtra,xunicode}
  \else
    \usepackage{fontspec}
  \fi
  \defaultfontfeatures{Mapping=tex-text,Scale=MatchLowercase}
  \newcommand{\euro}{€}
\fi
% use upquote if available, for straight quotes in verbatim environments
\IfFileExists{upquote.sty}{\usepackage{upquote}}{}
% use microtype if available
\IfFileExists{microtype.sty}{\usepackage{microtype}}{}

% Table formatting
\usepackage{longtable, booktabs}
\usepackage{lscape}
% \usepackage[counterclockwise]{rotating}   % Landscape page setup for large tables
\usepackage{multirow}		% Table styling
\usepackage{tabularx}		% Control Column width
\usepackage[flushleft]{threeparttable}	% Allows for three part tables with a specified notes section
\usepackage{threeparttablex}            % Lets threeparttable work with longtable

% Create new environments so endfloat can handle them
% \newenvironment{ltable}
%   {\begin{landscape}\begin{center}\begin{threeparttable}}
%   {\end{threeparttable}\end{center}\end{landscape}}

\newenvironment{lltable}
  {\begin{landscape}\begin{center}\begin{ThreePartTable}}
  {\end{ThreePartTable}\end{center}\end{landscape}}

  \usepackage{ifthen} % Only add declarations when endfloat package is loaded
  \ifthenelse{\equal{\string man}{\string man}}{%
   \DeclareDelayedFloatFlavor{ThreePartTable}{table} % Make endfloat play with longtable
   % \DeclareDelayedFloatFlavor{ltable}{table} % Make endfloat play with lscape
   \DeclareDelayedFloatFlavor{lltable}{table} % Make endfloat play with lscape & longtable
  }{}%



% The following enables adjusting longtable caption width to table width
% Solution found at http://golatex.de/longtable-mit-caption-so-breit-wie-die-tabelle-t15767.html
\makeatletter
\newcommand\LastLTentrywidth{1em}
\newlength\longtablewidth
\setlength{\longtablewidth}{1in}
\newcommand\getlongtablewidth{%
 \begingroup
  \ifcsname LT@\roman{LT@tables}\endcsname
  \global\longtablewidth=0pt
  \renewcommand\LT@entry[2]{\global\advance\longtablewidth by ##2\relax\gdef\LastLTentrywidth{##2}}%
  \@nameuse{LT@\roman{LT@tables}}%
  \fi
\endgroup}


\ifxetex
  \usepackage[setpagesize=false, % page size defined by xetex
              unicode=false, % unicode breaks when used with xetex
              xetex]{hyperref}
\else
  \usepackage[unicode=true]{hyperref}
\fi
\hypersetup{breaklinks=true,
            pdfauthor={},
            pdftitle={The effect of randomization on factor structure},
            colorlinks=true,
            citecolor=blue,
            urlcolor=blue,
            linkcolor=black,
            pdfborder={0 0 0}}
\urlstyle{same}  % don't use monospace font for urls

\setlength{\parindent}{0pt}
%\setlength{\parskip}{0pt plus 0pt minus 0pt}

\setlength{\emergencystretch}{3em}  % prevent overfull lines

\ifxetex
  \usepackage{polyglossia}
  \setmainlanguage{}
\else
  \usepackage[english]{babel}
\fi

% Manuscript styling
\captionsetup{font=singlespacing,justification=justified}
\usepackage{csquotes}
\usepackage{upgreek}

 % Line numbering
  \usepackage{lineno}
  \linenumbers


\usepackage{tikz} % Variable definition to generate author note

% fix for \tightlist problem in pandoc 1.14
\providecommand{\tightlist}{%
  \setlength{\itemsep}{0pt}\setlength{\parskip}{0pt}}

% Essential manuscript parts
  \title{The effect of randomization on factor structure}

  \shorttitle{Randomizing factors}


  \author{Erin M. Buchanan\textsuperscript{1}, David J. Herr\textsuperscript{1}, Becca N. Johnson\textsuperscript{1}, Hannah Myers\textsuperscript{1}, Jeffrey M. Pavlacic\textsuperscript{1}, \& Rachel Swadley\textsuperscript{1}}

  \def\affdep{{"", "", "", "", "", ""}}%
  \def\affcity{{"", "", "", "", "", ""}}%

  \affiliation{
    \vspace{0.5cm}
          \textsuperscript{1} Missouri State University  }

  \authornote{
    \newcounter{author}
    Complete departmental affiliations for each author (note the
    indentation, if you start a new paragraph). Enter author note here.

                      Correspondence concerning this article should be addressed to Erin M. Buchanan, 901 S. National Ave. E-mail: \href{mailto:erinbuchanan@missouristate.edu}{\nolinkurl{erinbuchanan@missouristate.edu}}
                                                                  }


  \abstract{Enter abstract here (note the indentation, if you start a new
paragraph).}
  \keywords{keywords \\

    \indent Word count: X
  }





\begin{document}

\maketitle

\setcounter{secnumdepth}{0}



(Buchanan, 2002) says that stuff is stuff and stuff. ({\textbf{???}})
said more stuff.

\section{Methods}\label{methods}

\subsection{Participants}\label{participants}

need someone to talk about the participants using the demographics file
(which needs to be cleaned up)

\subsection{Materials}\label{materials}

need someone to make a table of the scales we used

\subsection{Procedure}\label{procedure}

\subsection{Data analysis}\label{data-analysis}

\subsubsection{Data Screening}\label{data-screening}

\subsubsection{MGCFA}\label{mgcfa}

Multigroup Confirmatory Factor Analysis (MG-CFA) was conducted on
individual meaning in life scales. This particular process involves
applying CFA principles to multiple groups across different each
individual scale. Delivery type (non-random vs.~random) was used to
examine model fit and whether or not randomization of scales produces a
worse or better-fitting model. We utilized previously published
standards for adding restrictions to each MG-CFA. This approach allowed
us to first examine model fit across all groups. Subsequently, model fit
across non-random and random groups was examined. Then, parameters were
constrained in order to calculate different types of invariances.

\paragraph{Individual Groups}\label{individual-groups}

Utilizing a stepwise approach allowed us to examine model fit across
individual groups by means of MG-CFA. We conducted single-group
solutions based on delivery method (non-random question order vs.~random
question order). Questions delivered on paper were excluded for final
analysis in R, as they were not part of this particular analysis. Each
group provided us with a set of fit indices by which to evaluate model
fit and examine whether or not scale randomization impacts factor
structure across each different scale. Randomized scales not adhering to
the published factor structure should warrant caution among researchers
planning to deliver questions in a random format. Randomized scales
adhering to published factor structure do not suggest any reason to
avoid randomization of questions (Brown citation). Regardless of fit, we
continued with the suggested stepwise approach by calculating different
types of invariances. Each level of invariance adds restrictions to the
model.

\paragraph{Configural Invariance}\label{configural-invariance}

Regardless of whether or not our individualized groups both showed
adequate model fit, we progressed to calculate configural invariance.
Configural invariance can also be referred to as \enquote{equal form.}
This test allows the researcher to understand whether or not factor
structure and loadings are identical across groups, in this case
non-random questionnaires vs.~random questionnaires. This test utilizes
the same set of fit indices explained above (assuming we will add this
section in the data analysis section/insert a citatio).

\paragraph{Metric Invariance}\label{metric-invariance}

Regardless of whether or not equal forms was supported across groups, we
then analyzed the data using metric invariance. Metric invariance
examines factor loadings across groups. This analysis was supported if
this test of invariance did not differ significantly from configural
invariance. In order to meet this assumption, ∆CFI \textless{} .01.

\paragraph{Scalar Invariance}\label{scalar-invariance}

Assuming that metric invariance did provide a large enough decrease in
CFI, we then tested scalar invariance. Scalar invariance examines
indicator intercepts and determines whether or not these are equal
across groups. Additionally, scalar invariance determines whether or not
group membership influences a role in raw scores across groups. If the
change in CFI is not equal to or greater than .01, this assumption has
been met. As with metric invariance, this analysis was supported if the
test of invariance did not differ significantly from configural
invariance.

\paragraph{Partial Invariance}\label{partial-invariance}

Different methods have been utilized for scales that differ when
utilizing the stepwise method for conducting the different types of
invariances. The scale can either be abandoned or the noninvariant items
removed for further analyses. This may affect construct validity as well
as the theory behind the scale (Cheung \& Rensvold, 1999). We relaxes
constructs of noninvariant items for the remainder of analysis, as
suggested by Brown (2006) \& Byrne et al. (1989).

\section{Results}\label{results}

\section{Discussion}\label{discussion}

\newpage

\section{References}\label{references}

\setlength{\parindent}{-0.5in} \setlength{\leftskip}{0.5in}

\hypertarget{refs}{}
\hypertarget{ref-Buchanan2002}{}
Buchanan, T. (2002). Online assessment: Desirable or dangerous?
\emph{Professional Psychology: Research and Practice}, \emph{33}(2),
148--154.
doi:\href{https://doi.org/10.1037/0735-7028.33.2.148}{10.1037/0735-7028.33.2.148}






\end{document}
